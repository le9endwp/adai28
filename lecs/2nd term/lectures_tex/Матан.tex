\documentclass[12pt]{article}
\usepackage{config}
\usepackage{subfiles}

\def\multiset#1#2{\ensuremath{\left(\kern-.3em\left(\genfrac(){0pt}{}{#1}{#2}\right)\kern-.3em\right)}}
\def\divby{%
  \mathrel{\text{\vbox{\baselineskip.65ex\lineskiplimit0pt\hbox{.}\hbox{.}\hbox{.}}}}%
}
\newcommand{\q}[1]{\langle #1 \rangle}

\begin{document}

\begin{flushright}
    Конспект Шорохова Сергея

    Если нашли опечатку/ошибку - пишите @le9endwp
\end{flushright}

\subsection*{\S 1. Первообразная и неопределенный интеграл}

\begin{defin}{Первообразная функция}
    $f : \q{a, b} \rightarrow \R;\ \ F: \q{a, b} \rightarrow \R$ 

    $F$ -- первообразная функция $f$, если $F$ дифференцируема на $\q{a, b}$ и $F'(x) = f(x)$ при всех $x \in \q{a, b}$

    \begin{Example}{}
        $f(x) = \cos{x}$

        $F(x) = \sin{x}$ 
    \end{Example}
\end{defin}

\begin{propos}{}
    Не всякая функция имеет первообразную

    \begin{Example}{}
        $f(x) = \begin{cases}
            0,\ x \in (-1, 0] \\
            1,\ x \in (0, 1)
        \end{cases}$
    \end{Example}
\end{propos}

\begin{propos}{}
    Непрерывная на $\q{a, b}$ функция имеет первообразную
\end{propos}

\begin{theo}{}
    $f, F : \q{a, b} \rightarrow \R, F$ -- первообразная $f$. Тогда

    \begin{enumerate}
        \item $F + C$ -- первообразная $f$
        \item Если $\Phi$ -- первообразная $f$, то $\Phi = F + C$ для некоторой константы $C$
    \end{enumerate}
\end{theo}

\textit{Доказательство:}

\begin{enumerate}
    \item $(F + C)' = F' = f$
    \item $\Phi' = f = F'$
    
    $g = \Phi - F$

    $g' = 0 \Rightarrow g = C \Rightarrow \Phi = F + C$
\end{enumerate}

\begin{defin}{Неопределенный интеграл}
    Неопределенный интеграл -- множество первообразных функции $f$

    Обозначение: $\int f(x)dx$
\end{defin}

\begin{Remark}{}
    Для доказательства равенства $\int f(x)dx = F(x) + C$ достаточно проверить, что 
    
    $F'(x) = f(x)$
\end{Remark}

\textbf{Действия с множествами функций:}

$A$ и $B$ -- множества функций $\q{a, b} \rightarrow \R$

$\lambda \in \R,\ h : \q{a, b} \rightarrow \R$

\begin{enumerate}
    \item $A + B = \{f + g : f \in A,\ g \in B\}$
    \item $\lambda A = \{\lambda f : f \in A\}$
    \item $A + h = \{f + h : f \in A\}$
    \item $(A)' = \{f' : f \in A\}$
    
    \begin{Example}{}
        $(\int f(x)dx)' = \{ f \}$
    \end{Example}
\end{enumerate}

\textbf{Таблица интегралов:}

\begin{enumerate}
    \item $\int adx = ax + C$
    \item $\int x^pdx = \frac{x^{p + 1}}{p + 1} + C,\ p \neq -1$
    \item $\in \frac{dx}{x} = \ln{|x|} + C$
    \item $\int a^xdx = \frac{a^x}{\ln{a}} + C;\ a > 0,\ a \neq 1$
    \item $\int \sin{x}dx = -\cos{x} + C$
    \item $\int \cos{x}dx = \sin{x} + C$
    \item $\int \frac{dx}{x^2 + 1} = \arctg{x} + C$
    \item $\int \frac{dx}{\sqrt{1 - x^2}} = \arcsin{x} + C$
\end{enumerate}

\begin{theo}{Линейность интеграла}
    $f, g : \q{a, b} \Rightarrow \R$ имеют первообразные

    $\alpha, \beta \in \R$, не равные нулю одновременно

    Тогда $\int (\alpha f(x) + \beta g(x))dx = \alpha \int f(x)dx + \beta \int g(x)dx$
\end{theo}

\textit{Доказательство:}

$F$ и $G$ -- первообразные

Правая часть $= \{ \alpha F(x) + \beta G(x) + C : C \in \R \}$

$(\alpha F(x) + \beta G(x) + C)' = \alpha F'(x) + \beta G'(x) = \alpha f(x) + \beta g(x)$

\begin{theo}{Замена переменной в интеграле}
    $f : \q{a, b} \rightarrow \R,\ F$ -- первообразная
    
    $\varphi : \q{c, d} \rightarrow \q{a, b}$ -- дифференцируемая функция

    Тогда $\int f(\varphi(x))\varphi'(x)dx = F(\varphi(x)) + C$
\end{theo}

\textit{Доказательство:}

$(F(\varphi(x)) + C)' = F'(\varphi(x))\varphi'(x) = f(\varphi(x))\varphi'(x)$

\begin{Remark}{}
    $y = \varphi(x);\ \ dy = \varphi'(x)dx$

    $\frac{dy}{dx} = y'$

    $\int f(\varphi(x)) \varphi'(x)dx = \int f(y)dy = F(y) + C = F(\varphi(x)) + C$
\end{Remark}

\begin{Example}{}
    \begin{enumerate}
        \item $\int \frac{x}{x^2 + 1}dx = \frac{1}{2} \int \frac{(x^2 + 1)'}{x^2 + 1}dx = \frac{1}{2} \int \frac{dy}{y} = \ln{|y|} + c = \ln{|x^2 + 1|} + C$
        
        Здесь $y = \varphi(x) = x^2 + 1$

        \item $\int \frac{dx}{\sin{x}} = \int \frac{dx}{2\sin{\frac{x}{2}}\cos{\frac{x}{2}}} = \int \frac{dy}{\sin{y}\cos{y}} = \int \frac{dy}{\tg{y}\cos^2{y}} = \int \frac{(\tg{y})'}{tg{y}}dy = \int \frac{dz}{z} = \ln{|z|} + C =$
        
        $= \ln{|\tg{y}|} + C = \ln{|\tg{\frac{x}{2}}|} + C$

        Здесь $y = \frac{x}{2}$ и $z = \tg{y}$

        \item $\int \frac{dx}{1 + \sqrt[3]{x}} = \int \frac{3t^2dt}{1 + t} = 3\int \frac{t^2 - 1 + 1}{t + 1}dt = 3\int (t - 1 + \frac{1}{t + 1})dt = 3(\int tdt - \int dt + \int \frac{dt}{t + 1}) =$
        
        $= 3t^2 - 3t + 3\int \frac{d(t + 1)}{t + 1} = 3t^2 - 3t + 3\ln{|t + 1|} + C$
    \end{enumerate}
\end{Example}

\begin{theo}{Интегрирование по частям}
    $f, g : \q{a, b} \rightarrow \R$ дифференцируемые

    Если $f'g$ имеет первообразную, то $\int f(x)g'(x)dx = f(x)g(x) - \int f'(x)g(x)dx$
\end{theo}

\textit{Доказательство:}

$H$ -- первообразная функции $f'g$

$(fg - H + C)' = (fg)' - H' = f'g + fg' - f'g = fg'$

\begin{nota}{Традиционная запись формулы}
    $\int udv = uv - \int vdu$

    $\begin{cases}
        du = u'(x)dx \\
        dv = v'(x)dx
    \end{cases}$
\end{nota}

\begin{Example}{}
    \begin{enumerate}
        \item $\int \ln{x}dx = x\ln{x} - \int x \frac{dx}{x} = x\ln{x} - \int dx = x\ln{x} - x + C$
        
        Здесь $u = \ln{x},\ v = x$ и $du = (\ln{x})'dx = \frac{dx}{x}$ 

        \item $\int x^2 e^x dx = \int x^2 de^x = x^2 e^x - \int 2x e^x dx = x^2 e^x - 2\int xde^x = x^2e^x - 2(xe^x - \int e^xdx) = \\ = x^2e^x - 2xe^x + 2e^x + C$
        
        Здесь сначала берем $u = x^2, v = e^x$, а потом $u = x, v = e^x$
    \end{enumerate}
\end{Example}

\subsection*{\S 2. Площадь}

\begin{defin}{Площадь}
    $F$ -- семейство всех ограниченных подмножеств плоскости

    Прямоугольник $\q{a_1, b_1} \times \q{a_2, b_2}$, площадь прямоугольника $(b_1 - a_1)(b_2 - a_2)$

    Площадь $S : F \rightarrow [0, + \infty)$

    \begin{enumerate}
        \item $S(\q{a_1, b_1} \times \q{a_2, b_2}) = (b_1 - a_1)(b_2 - a_2)$
        \item $S(E) = S(E_1) + S(E_2)$, если $E = E_1 \bigcup E_2,\ E_1 \bigcap E_2 = \varnothing$
    \end{enumerate}
\end{defin}

\begin{theo}{Свойство}
    Если $\tilde{E} \subset E$, то $S(\tilde{E}) \leq S(E)$
\end{theo}

\textit{Доказательство:}

$E = \tilde{E} \bigcup (E \setminus \tilde{E})$

$S(E) = S(\tilde{E}) + S(E \setminus \tilde{E}) \geq S(\tilde{E})$

\begin{defin}{(Квази)площадь}
    $\sigma : F \rightarrow [0, +\infty)$

    \begin{enumerate}
        \item $\sigma(\q{a_1, b_1} \times \q{a_2, b_2}) = (b_1 - a_1)(b_2 - a_2)$
        \item $\sigma(E) = \sigma(E_-) + \sigma(E_+)$, если $E_-$ и $E_+$ множества, получающиеся в результате разбиения $E$ вертикальной (горизонтальной) прямой
        \item Если $\tilde{E} \subset E$, то $\sigma(\tilde{E}) \leq \sigma(E)$
    \end{enumerate}
\end{defin}

\begin{Remark}{Свойство}
    Формула 2) верна и если $E_- \bigcap E_+ \neq \varnothing$

    Например, линию разбиения можно считать относящейся и к левой (верхней), и к правой (нижней) части
\end{Remark}

\textit{Доказательство:}

$e = E_- \bigcap E_+$, $\sigma(e) = 0$

$\sigma(E_+) = \sigma(E_+ \setminus e) + \sigma(e \bigcap E_+) = \sigma(E_+ \setminus e)$

$\sigma(E_-) + \sigma(E_+) = \sigma(E_-) + \sigma(E_+ \setminus e) = \sigma(E_- \bigcup (E_+ \setminus e)) = \sigma(E_- \bigcup E_+) = \sigma(E)$

\begin{Example}{Примеры площадей $E \in F$}
    \begin{itemize}
        \item Рассмотрим покрытие $E$ конечным числом прямоугольников $P_i$ (т.е. $\bigcup\limits_{i = 1}^n P_i \supset E$)
        
        $\sigma_1(E) = \inf\{ \sum\limits_{i = 1}^n \sigma(P_i) : \bigcup\limits_{i = 1}^n P_i \supset E \}$

        \item Рассмотрим покрытие $E$ последовательностью прямоугольников $P_i$ (т.е. $\bigcup\limits_{i = 1}^{\infty} P_i \supset E$)
        
        $\sigma_2(E) = \inf\{ \sum\limits_{i = 1}^{\infty} \sigma(P_i) : \bigcup\limits_{i = 1}^{\infty} P_i \supset E \}$

        \item Ясно, что $\sigma_1(E) \geq \sigma_2(E)$
        
        Но, если $E = ([0, 1] \bigcap \Q) \times ([0, 1] \bigcap \Q)$, то $\begin{cases}
            \sigma_1(E) = 1 \\
            \sigma_2(E) = 0
        \end{cases}$
    \end{itemize}
\end{Example}

\begin{theo}{}
    \begin{enumerate}
        \item $\sigma_1$ -- площадь
        \item $\sigma_1$ не меняется при параллельном переносе
    \end{enumerate}
\end{theo}

\textit{Доказательство:}

\textbf{1)}

\begin{enumerate}
    \item $\sigma_1(\q{a, b} \times \q{c, d}) = (b - a)(d - c)$
    
    Поскольку $[a, b] \times [c, d]$ -- покрытие $P$, $\sigma_1(P) \leq (b - a)(d - c)$

    В обратную сторону красиво доказано АИ. Там рисуночки, посмотрите!

    \item $E = E_- \bigcup E_+ \Rightarrow \sigma_1(E) = \sigma_1(E_-) + \sigma_1(E_+)$
    
    \begin{itemize}
        \item[$\leq :$] Если $P_1^+, \ldots P_m^+$ -- покрытие $E_+$, для которого $\sum\limits_{i = 1}^m \sigma(P_i^+) < \sigma_1(E_+) + \eps$
        
        А $P_1^-, \ldots P_n^-$ -- покрытие $E_-$, для которого $\sum\limits_{i = 1}^n \sigma(P_i^-) < \sigma_1(E_-) + \eps$, то

        $P_1^-, P_2^-, \ldots P_n^-, P_1^+, P_2^+, \ldots P_m^+$ -- покрытие $E$, для которого 
        
        $\sigma_1(E) \leq \sum\limits_{i = 1}^{n + m} \sigma(P_i) < \sigma_1(E_-) + \sigma_1(E_+) + 2\eps \Rightarrow \sigma_1(E) < \sigma_1(E_-) + \sigma_1(E_+) + 2\eps$

        \item[$\geq :$] Пусть $P_1, P_2, \ldots P_n$ -- покрытие $E$
        
        Разобьем $P_i$ на $P_i^-$ и $P_i^+$

        $\sigma(P_i) = \sigma(P_i^-) + \sigma(P_i^+)$

        $P_1^\pm, P_2^\pm, \ldots P_n^\pm$ -- покрытие $E^\pm$

        $\sum\limits_{i = 1}^n \sigma(P_i^\pm) \geq \sigma_1(E^\pm)$

        $\sum\limits_{i = 1}^n (\sigma(P_i^-) + \sigma(P_i^+)) \geq \sigma_1(E_-) + \sigma_1(E_+)$
    \end{itemize}

    \item $\tilde{E} \subset E \Rightarrow \sigma_1(\tilde{E}) \leq \sigma_1(E)$
    
    Если $\bigcup\limits_{i = 1}^n P_i \supset E$, то $\bigcup\limits_{i = 1}^n P_i \supset \tilde{E} \Rightarrow$ класс покрытий $\tilde{E}$ шире, чем класс покрытий $E$
\end{enumerate}

\textbf{2)}

Пусть $\tilde{E}$ -- параллельный перенос $E$ на вектор $\overrightarrow{v}$

$P_1, P_2, \ldots P_n$ -- покрытие $E$. Пусть $\tilde{P_i}$ -- параллельный перенос $P_i$ на вектор $\overrightarrow{v}$

Тогда $\tilde{P_1}, \tilde{P_2}, \ldots \tilde{P_n}$ -- покрытие $\tilde{E}$ и $\sum\limits_{i = 1}^n \sigma(P_i) = \sum\limits_{i = 1}^n \sigma(\tilde{P_i})$

\begin{defin}{}
    $f : [a, b] \rightarrow \R$

    $f_+ := \max\{f, 0\}$, т.е. $f_+(x) = \max\{f(x), 0\}$

    $f_- := \max\{-f, 0\}$, т.е. $f_-(x) = \max\{-f(x), 0\}$

    Свойства:

    \begin{enumerate}
        \item $f_\pm \geq 0$
        \item $f = f_+ - f_-$
        
        $|f| = f_+ + f_-$

        \item $f_+ = \frac{f + |f|}{2}$ и $f_- = \frac{|f| - f}{2}$
        \item Если $f \in C[a, b]$, то $f_\pm \in C[a, b]$
    \end{enumerate}
\end{defin}

\begin{defin}{Подграфик функции}
    $f : [a, b] \rightarrow \R,\ f \geq 0$

    Подграфик функции $f$ -- $P_f = \{ (x, y) \in \R : x \in [a, b],\ 0 \leq y \leq f(x) \}$
\end{defin}

\begin{defin}{Определенный интеграл}
    $\sigma$ -- зафиксированная квазиплощадь

    $f \in C[a, b]$ (пока что так)

    $\int\limits_a^b f = \int\limits_a^b f(x)dx  := \sigma(P_{f_+}) - \sigma(P_{f_-})$
\end{defin}

\textbf{Свойства:}

    \begin{enumerate}
        \item $\int\limits_a^a f = 0$
        \item $\int\limits_a^b 0 = 0$
        \item Если $f \geq 0$, то $\int\limits_a^b f = \sigma(P_f)$
        \item $\int\limits_a^b (-f) = -\int\limits_a^b f$
        
        \textit{Доказательство:}

        $(-f_+) = \max\{-f, 0\} = f_-$

        $(-f_-) = \max\{f, 0\} = f_+$

        $\int\limits_a^b (-f) = \sigma(P_{f_-}) - \sigma(P_{f_+}) = -\int\limits_a^b f$

        \item $\int\limits_a^b (c) = c(b - a)$
        
        \textit{Доказательство:}

        $c > 0 \Rightarrow \int\limits_a^b c = P(\text{прямоугольника}) = c(b - a)$

        \item Если $a < b,\ f \geq 0$ и $\int\limits_a^b f = 0$, то $f \equiv 0$
        
        \textit{Доказательство:} (от противного)

        Пусть $f(x_0) > 0$. Из непрерывности $f$ в $x_0 \Rightarrow \varepsilon = \frac{f(x_0)}{2} \Rightarrow \exists \delta > 0 : \forall |x - x_0| < \delta \Rightarrow \\
        \Rightarrow |f(x) - f(x_0)| < \varepsilon = \frac{f(x_0)}{2} \Rightarrow P_f \supset [x_0 - \delta, x_0 + \delta] \times [0, \frac{f(x_0)}{2}] \Rightarrow \\
        \Rightarrow \sigma(P_f) \geq \sigma(\text{прямоугольника}) = 2\sigma \frac{f(x_0)}{2} > 0$ Противоречие
    \end{enumerate}

\subsection*{\S 3. Свойства интеграла}

\begin{nota}{Обозначение}
    $P_g(E)$ -- подграфик функции $g \geq 0$ над множество $E$, т.е. 

    $P_g(E) := \{ (x, y) \in \R^2 : x \in E,\ 0 \leq y \leq g(x) \}$
\end{nota}

\begin{theo}{Аддитивность интеграла}
    $f \in C[a, b]$ и $c \in [a, b]$

    Тогда $\int\limits_a^b f = \int\limits_a^c f + \int\limits_c^b f$
\end{theo}

\textit{Доказательство:}

$\int\limits_a^b f = \sigma(P_{f_+}) - \sigma(P_{f_-}) = \sigma(P_{f_+}([a, c])) + \sigma(P_{f_+}([c, b])) - \sigma(P_{f_-}([a, c])) - \sigma(P_{f_-}([c, b])) = \int\limits_a^c f + \int\limits_c^b f$

\begin{theo}{Следствие}
    $f \in C[a, b]$, $a \leq c_1 \leq c_2 \leq \ldots \leq c_n \leq b$. Тогда

    $\int\limits_a^b f = \int\limits_a^{c_1} f + \int\limits_{c_1}^{c_2} f + \ldots + \int\limits_{c_{n - 1}}^{c_n} f + \int\limits_{c_n}^b f$
\end{theo}

\textit{Доказательство:}

Индукция по $n$

\begin{theo}{Монотонность интеграла}
    $f, g \in C[a, b]$ и $f \leq g$ на $[a, b]$

    Тогда $\int\limits_a^b f \leq \int\limits_a^b g$
\end{theo}

\textit{Доказательство:}

$f \leq g \Rightarrow f_+ \leq g_+ \Rightarrow P_{f_+} \subset P_{g_+}$, а еще $-g \leq -f \Rightarrow g_- \leq f_- \Rightarrow P_{g_-} \subset P_{f_-}$

Значит $\sigma (P_{f_+}) \leq \sigma (P_{g_+})$ и $\sigma (P_{g_-}) \leq \sigma (P_{f_-})$

$\int\limits_a^b f = \sigma (P_{f_+}) - \sigma (P_{f_-}) \leq \sigma (P_{g_+}) - \sigma (P_{g_-}) = \int\limits_a^b g$

\begin{theo}{Следствия}
    \begin{enumerate}
        \item $f \in C[a, b] \Rightarrow \min\limits_{[a, b]} f \cdot (b - a) \leq \int\limits_a^b f \leq \max\limits_{[a, b]} f \cdot (b - a)$
        
        \textit{Доказательство:}

        $\min f \leq f \leq \max f$ и монотонность интеграла для двух постоянных функций и $f$

        \item $f \in C[a, b] \Rightarrow |\int\limits_a^b f| \leq \int\limits_a^b |f|$
        
        \textit{Доказательство:}

        $-|f| \leq f \leq |f| \xRightarrow{\text{монотонность}} -\int |f| = \int\limits_a^b (-|f|) \leq \int\limits_a^b f \leq \int\limits_a^b |f|$
    \end{enumerate}
\end{theo}

\begin{theo}{(Первая) (интегральная) теорема о среднем}
    $f \in C[a, b]$. Тогда существует $c \in [a, b] : \int\limits_a^b f = f(c)(b - a)$
\end{theo}

\textit{Доказательство:}

$\min f \leq \frac{1}{b - a} \int\limits_a^b f \leq \max f$, но множество значений $f$ на $[a, b]$ -- это отрезок $[\min f, \max f]$

Следовательно, число $\frac{1}{b - a} \int\limits_a^b f$ -- есть значение функции $f$ в какой-то точке $[a, b]$. Возьмем эту точку в качестве $c$

\begin{defin}{Среднее значение функции на отрезке}
    Среднее значение функции $f$ на отрезке $[a, b]$ -- это $\frac{1}{b - a} \int\limits_a^b f$
\end{defin}

\begin{defin}{Интеграл с переменным верхним пределом}
    $f \in C[a, b]$

    $\Phi(x) := \int\limits_a^x f$, где $x \in [a, b]$
\end{defin}

\begin{Remark}{}
    $\Phi(a) = 0$
\end{Remark}

\begin{defin}{Интеграл с переменным нижним пределом}
    $f \in C[a, b]$

    $\Psi(x) := \int\limits_x^b f$, где $x \in [a, b]$
\end{defin}

\begin{Remark}{}
    $\Psi(b) = 0$

    $\Phi(x) + \Psi(x) = \int\limits_a^b f$ (это аддитивность $\int$)
\end{Remark}

\begin{theo}{Теорема Барроу}
    Если $f \in C[a, b],\ \Phi(x) := \int\limits_a^x f$, то $\Phi$ -- первообразная функции $f$
\end{theo}

\textit{Доказательство:}

Надо доказать, что $\Phi'(x) = f(x)$. Пусть $x < y$

$R(y) = \frac{\Phi(y) - \Phi(x)}{y - x} = \frac{1}{y - x} (\int\limits_a^y f - \int\limits_a^x f) = \frac{1}{y - x} \int\limits_x^y f \stackrel{\text{т-ма о среднем}}{=} f(c_y)$, где $c_y \in [x, y]$

Возьмем последовательность $y_n > x$ и $\lim y_n = x$

$\Phi_+'(x) = \lim\limits_{y \to x_+} R(y) = \lim\limits_{n \to \infty} R(y_n) = \lim\limits_{n \to \infty} f(c_{y_n}) = f(x)$, т.к. $x \leq c_{y_n} \leq y_n \rightarrow x$

Если же $y < x$, то нужно смотреть на $\frac{1}{x - y} \int\limits_y^x f$ и дальше ровно так же 

Следовательно, $\Phi'(x) = f(x)$

\begin{theo}{Следствия}
    \begin{enumerate}
        \item $\Psi(x) := \int\limits_x^b f \Rightarrow \Psi'(x) = -f(x)$
        
        \textit{Доказательство:}

        $\Phi(x) + \Psi(x) = const$

        \item Если $f \in C\q{a, b}$, то у $f$ есть первообразная на $\q{a, b}$
        
        \textit{Доказательство:}

        Возьмем $c \in (a, b)$ и $F(x) := \begin{cases}
            \int\limits_c^x f,\ x \geq c \\
            -\int\limits_x^c f,\ x \leq c
        \end{cases}$
        
        Тогда $\begin{gathered}
            F'(x) = f(x) \text{при} x \geq c \text{ (по теореме Барроу)} \\
            F'(x) = -f(x) \text{при} x \leq c \text{ (по следствию 1)} \\
            F_+'(c) = f(c) = F_-'(c)
        \end{gathered}$
    \end{enumerate}
\end{theo}

\begin{theo}{Формула Ньютона-Лейбница}
    $f \in C[a, b],\ F$ -- первообразная $f$

    Тогда $\int\limits_a^b f = F(b) - F(a)$
\end{theo}

\textit{Доказательство:}

$\Phi(x) := \int\limits_a^x f$ -- первообразная $f$ (по теореме Барроу) $\Rightarrow \Phi = F + C$ для некоторой $C \in \R$

$\Rightarrow \int\limits_a^b f = \Phi(b) = F(b) + C = F(b) - F(a)$, т.к. $0 = \Phi(a) = F(a) + C$

\begin{nota}{Обозначение}
    $F|_a^b := F(b) - F(a)$ подстановка

    $\int\limits_a^b f = F|_a^b$
\end{nota}

\begin{theo}{Линейность интеграла}
    $f, g \in C[a, b],\ \alpha, \beta \in \R$

    Тогда $\int\limits_a^b (\alpha f + \beta g) = \alpha \int\limits_a^b f + \beta \int\limits_a^b g$
\end{theo}

\textit{Доказательство:}

Пусть $F$ и $G$ -- первообразные $f$ и $g$

Тогда $\alpha F + \beta G$ -- первообразная $\alpha f + \beta g \Rightarrow \int\limits_a^b (\alpha f + \beta g) = (\alpha F + \beta G)|_a^b = \alpha F|_a^b + \beta G|_a^b = \\
= \alpha \int\limits_a^b f + \beta \int\limits_a^b g$

\begin{theo}{Формула интегрирования по частям}
    $u, v \in C^1[a, b]$

    Тогда $\int\limits_a^b uv' = uv|_a^b - \int\limits_a^b u'v$
\end{theo}

\textit{Доказательство:}

Пусть $H$ -- первообразная $u'v$. Тогда $uv - H$ -- первообразная $uv'$

$(uv - H)' = u'v + uv' - H' = u'v + uv' - u'v = uv'$

$\int\limits_a^b uv' = (uv - H)|_a^b = uv|_a^b - H|_a^b = uv|_a^b - \int\limits_a^b u'v$

\begin{nota}{Соглашение}
    Если $a > b$, то $\int\limits_a^b f = -\int\limits_b^a f$
\end{nota}

\begin{theo}{Замена переменной в определенном интеграле}
    $f \in C\q{a, b},\ \varphi \in C^1\q{c, d},\ \varphi : \q{c, d} \rightarrow \q{a, b},\ p, q \in \q{c, d}$. Тогда

    $\int\limits_p^q f(\varphi(t))\varphi'(t)dtt = \int\limits_{\varphi(p)}^{\varphi(q)} f(x)dx$
\end{theo}

\textit{Доказательство:}

Пусть $F$ -- первообразная для $f$. Тогда $F \circ \varphi$ -- первообразная для $f(\varphi(t))\varphi'(t)$ (т.к. $(F(\varphi(t)))' = F'(\varphi(t))\varphi'(t)$)

$\int\limits_p^q f(\varphi(t))\varphi'(t)dt = (F \circ \varphi)|_p^q = F(\varphi(q)) - F(\varphi(p)) = \int\limits_{\varphi(p)}^{\varphi(q)} f$

\subsection*{\S 4. Приложение формулы интегрирования по частям}

$W_n := \int\limits_0^{\frac{\pi}{2}} \cos^n{x}dx \stackrel{(*)}{=} \int\limits_0^{\frac{\pi}{2}} \sin^n{x}dx$

Пояснение к $(*)$: $\int\limits_0^\frac{\pi}{2} \sin^n{x}ds = \int\limits_0^\frac{\pi}{2} \cos^n{\frac{\pi}{2} - t}dt = - \int\limits_{\varphi(0)}^{\varphi(\frac{\pi}{2})} \cos^n{x}dx = -\int\limits_0^{\frac{\pi}{2}} \cos^n{x}dx$

Здесь $\varphi(t) = \frac{\pi}{2} - t$ и $\varphi'(t) = -1$

$W_0 = \frac{\pi}{2},\ W_1 = \int\limits_0^{\frac{\pi}{2}} \sin{x}dx = -\cos{x}|_0^{\frac{\pi}{2}} = 1,\ W_2 = \frac{1}{2} (\int\limits_0^\frac{\pi}{2} \cos^2 + \int\limits_0^\frac{\pi}{2} \sin^2) = \frac{\pi}{4}$

$W_n = \int\limits_0^\frac{\pi}{2} \sin^n{x}dx = \int\limits_0^\frac{\pi}{2} \sin^{n-1}x \cdot (-\cos{x})'dx = -\sin^{n-1}x \cos{x}|_0^\frac{\pi}{2} - \int\limits_0^\frac{\pi}{2} (n-1)\sin^{n-2}x \cos{x} (-\cos{x})dx = \\
= (n - 1) (\int\limits_0^\frac{\pi}{2} \sin^{n-2}x (1 - \sin^2{x})dx - \int\limits_0^\frac{\pi}{2} \sin^n{x}dx) = (n - 1)(W_{n-2} - W_n) \Rightarrow nW_n = (n - 1)W_{n-2}$

Если четно, то $W_{2n} = \frac{2n - 1}{2n}W_{2n - 2} = \frac{2n - 1}{2n} \cdot \frac{2n - 3}{2n - 2} W_{2n - 1} = \ldots = \frac{(2n - 1)(2n - 3) \ldots 1}{2n(2n - 2) \ldots 2} \cdot \frac{\pi}{2} = \frac{(2n - 1)!!}{2n!!} \cdot \frac{\pi}{2}$

Если нечетно, то $W_{2n + 1} = \frac{2n}{2n + 1}W_{2n - 1} = \frac{2n}{2n + 1} \cdot \frac{2n - 2}{2n - 1} W_{2n - 3} = \ldots = \frac{2n(2n - 2) \ldots 2}{(2n + 1)(2n - 1) \ldots 3} \cdot 1 = \frac{2n!!}{(2n + 1)!!}$

\begin{theo}{Формула Валлеса}
    $\lim\limits_{n \to \infty} \frac{2n!!}{(2n + 1)!!} \cdot \frac{1}{\sqrt{2n + 1}} = \sqrt{\frac{\pi}{2}}$
\end{theo}

\textit{Доказательство:}

$\sin^{2n + 2}x \leq \sin^{2n + 1}x \leq \sin^{2n}x$ при $x \in [0, \frac{\pi}{2}]$

$\int\limits_0^\frac{\pi}{2} \sin^{2n + 2}x dx \leq \int\limits_0^\frac{\pi}{2} \sin^{2n + 1}x dx \leq \int\limits_0^\frac{\pi}{2} \sin^{2n}x dx$

То есть $W_{2n + 2} \leq W_{2n + 1} \leq W_{2n}$

$\frac{2n + 1}{2n + 2} \cdot \frac{(2n - 1)!!}{(2n)!!} \cdot \frac{\pi}{2} \leq \frac{2n!!}{(2n + 1)!!} \leq \frac{(2n - 1)!!}{(2n)!!} \cdot \frac{\pi}{2}$

$\frac{2n + 1}{2n + 2} \cdot \frac{\pi}{2} \leq \frac{(2n)!!}{(2n + 1)!!} \cdot \frac{(2n)!!}{(2n - 1)!!} \leq \frac{\pi}{2} \Rightarrow \lim (\frac{(2n)!!}{(2n - 1)!!})^2 \cdot \frac{1}{2n + 1} = \frac{\pi}{2}$

\end{document}