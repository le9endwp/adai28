\documentclass[12pt]{article}
\usepackage{config}
\usepackage{subfiles}
\pgfplotsset{compat=1.18}

\begin{document}

\begin{flushright}
    Конспект Шорохова Сергея

    Если нашли опечатку/ошибку - пишите @le9endwp 
\end{flushright}

\tableofcontents
\newpage

\section{Оргинфа}

Ведет Крыжевич Сергей Геннадьевич

+79219181076 и +48572768176

kryzhevicz@gmail.com и serkryzh@pg.edu.pl

\newpage

\section{Лекция 1. Дифференциальные уравнения первого порядка}

\begin{defin}{Дифференциальные уравнения первого порядка}
    $D \subset \R^2$ -- область, $f : D \to \R$ -- непрерывная функция

    Дифференциальные уравнения первого порядка -- это уравнения вида $y' = f(x, y)$
\end{defin}

\begin{Example}{}
    $y' = xy$
\end{Example}

\begin{defin}{Решение дифференциального уравнения}
    $\q{a, b}$ -- интервал

    Функция $\varphi(x)$ -- решение дифференциального уравнения на $\q{a, b}$, если 

    \begin{enumerate}
        \item $\varphi, \varphi'$ -- непрерывны на $\q{a, b}$
        \item $(x, \varphi(x)) \in D\ \forall x \in \q{a, b}$
        \item $\varphi'(x) = f(x, \varphi(x))$
    \end{enumerate}
\end{defin}

\begin{Example}{}
    $y' = xy$

    Решениями будут:

    \begin{enumerate}
        \item $y = 0$
        \item $y = e^{\frac{x^2}{2}}$
        
        $y' = xe^{\frac{x^2}{2}} = xy$
    \end{enumerate}

    На самом деле решением будет любая функция вида $y = Ce^{\frac{x^2}{2}}$
\end{Example}

\begin{nota}{Начальные данные для дифференциального уравнения}
    $\begin{cases}
        y' = f(x, y) \\
        y(x_0) = y_0
    \end{cases}$
\end{nota}

\begin{defin}{Задача Коши}
    Задача Коши -- дифференциальное уравнение с начальными данными
\end{defin}

\begin{Example}{}
    $\begin{cases}
        y' = xy \\
        y(0) = 5
    \end{cases}$

    $y = Ce^{\frac{x^2}{2}}$

    $5 = Ce^0 = C$

    Получаем ответ $y = 5e^{\frac{x^2}{2}}$
\end{Example}

\begin{defin}{Общее решение дифференциального уравнения}
    Общее решение дифференциального уравнения -- совокупность всех его решений (= решение с параметром)
\end{defin}

\begin{defin}{Интегральная кривая}
    Интегральная кривая -- график решения дифференциального уравнения, т.е. график $\{x, \varphi(x)\}$
\end{defin}

\begin{Remark}{}
    $y' = \sqrt{y};\ y \geq 0$

    Здесь множество не является открытым, но считается, что $y = 0$ является решением (хотя формально им не является)

    Если в каких-то задачах такое будет, в рамках курса не считаем это ошибкой
\end{Remark}

\begin{Remark}{Единственность решений задачи Коши}
    Почти всегда задача Коши имеет единственное решение. Но есть исключения, например

    $\begin{cases}
        y' = 3y^{\frac{2}{3}} \\
        y(0) = 0
    \end{cases}$

    Очевидное решение $y = 0$, но также $y = x^3$. Более того, решением будет любая функция вида $y = (x + C)^3$. График есть на записи

    Более того, можно собрать решение покусочно (ветка параболки вниз + прямая $y = 0$ + ветка параболы вверх)
\end{Remark}

\begin{defin}{Точка единственности/ветвления}
    $\begin{cases}
        y' = f(x, y) \\
        y(x_0) = y_0
    \end{cases}$

    Точка $(x_0, y_0)$ -- точка единственности, если решение задачи Коши единственно. В противном случае это точка ветвления 
\end{defin}

\begin{defin}{Особое решение}
    Решение называется особым, если любая его точка -- точка ветвления
\end{defin}

\begin{theo}{}
    Если в уравнении $y' = f(x, y)$ функция $f$ непрерывна и имеет непрерывную производную по переменной $y$ в области $D$, то для любой точки $(x_0, y_0)$ из $D$ решение задачи Коши с начальными данными $y(x_0) = y_0$ существует и единственно
\end{theo}

\begin{Remark}{}
    По $x$ нужна только непрерывность, производной существовать не обязательно
\end{Remark}

\begin{defin}{Дифференциальные уравнения в симметричной форме}
    $P(x, y)dx + Q(x, y)dy = 0$
\end{defin}

\begin{Example}{}
    $ydx - xdy = 0 \mapsto y' = \frac{y}{x}$ или $x' = \frac{x}{y}$
\end{Example}

\begin{Remark}{}
    Предполагаем, что $P$ и $Q$ -- функции, непрерывные в некоторой области $D$ на плоскости и они не обращаются в ноль одновременно ни в одной точке $D$
\end{Remark}

\begin{defin}{Решение уравнения в симметричной форме}
    \begin{enumerate}
        \item $y' = -\frac{P(x, y)}{Q(x, y)}$, решением будет $y = \varphi(x) : P(x, \varphi(x)) + Q(x, \varphi(x))\varphi'(x) = 0$
        \item $x' = -\frac{Q(x, y)}{P(x, y)}$, решением будет $x = \psi(y) : P(\psi(y), y)\psi'(y) + Q(\psi(y), y) = 0$
        \item $y = \varphi(t), x = \psi(t)$, хотим $P(\psi(t), \varphi(t))\psi'(t) + Q(\psi(t), \varphi(t))\varphi'(t) = 0$
    \end{enumerate}
\end{defin}

\newpage

\section{Лекция 2. Системы обыкновенных дифференциальных уравнений}

\begin{defin}{Системы обыкновенных дифференциальных уравнений}
    $t, x_1, \ldots, x_n \in \R;\ t$ -- время, $x_1 \ldots x_n$ -- фазовые переменные

    $\begin{cases}
        x_1 = f_1(t, x_1, \ldots, x_n) \\
        \ldots \\
        x_n = f_n(t, x_1, \ldots, x_n)
    \end{cases}$ -- скалярная запись системы

    $x = \begin{pmatrix}
        x_1 \\
        \ldots \\
        x_n
    \end{pmatrix};\ f = \begin{pmatrix}
        f_1 \\
        \ldots \\
        f_n
    \end{pmatrix}$ -- векторная запись системы
\end{defin}

\begin{nota}{Как свести уравнение высшего порядка к системам?}
    Пусть есть уравнение $x^{(n)} = g(t, x, x', \ldots, x^{(n-1)})$

    Полагаем $x_1 = x, \ldots, x_n = x^{(n-1)}$

    Получаем $\begin{cases}
        x_1' = x_2 \\
        \ldots \\
        x_{n-1}' = x_n \\
        x_n' = g(t, x_1, \ldots, x_n)
    \end{cases}$
\end{nota}

\begin{Example}{}
    $x'' + \sin x = 0$

    $\begin{cases}
        x_1' = x_2 \\
        x_2' = -\sin x_1
    \end{cases}$
\end{Example}

\begin{Remark}{}
    Предполагается что $f : D \to \R^n$ -- непрерывна и $D \subset \R^{n + 1}$
\end{Remark}

\begin{defin}{Решение системы}
    Функция $\varphi : \q{\alpha, \beta} \to \R^n$ называется решением системы если 

    \begin{enumerate}
        \item $\varphi \in C^1$
        \item $(t, \varphi(t)) \in D\ \forall t \in \q{\alpha, \beta}$
        \item $\varphi(t) = f(t, \varphi(t))\ \forall t \in \q{\alpha, \beta}$
    \end{enumerate}
\end{defin}

\begin{defin}{Задача Коши для систем}
    Пусть $t_0, x_{01}, \ldots, x_{0n} \in \R;\ (t_0, x_{01}, \ldots, x_{0n}) \in D$

    Начальные условия: $\begin{cases}
        x_1(t_0) = x_{01} \\
        \ldots \\
        x_n(t_0) = x_{0n}
    \end{cases}$

    Или в векторной форме: $x(t_0) = x_0$, где $x_0 = \begin{pmatrix}
        x_{01} \\
        \ldots \\
        x_{0n}
    \end{pmatrix}$

    Задача Коши -- уравнение + начальные условия

    $\begin{cases}
        x' = f(t, x) \\
        x(t_0) = x_0
    \end{cases}$
\end{defin}

\begin{defin}{Эквивалентное интегральное уравнение}
    $x(t) = x_0 + \int\limits_{t_0}^t f(x, x(s))ds$

    Функция $\varphi : \q{\alpha, \beta} \to \R^n$ называется решением эквивалентного интегрального уравнения, если

    \begin{enumerate}
        \item $\varphi$ -- непрерывна
        \item $(t, \varphi(t)) \in D\ \forall t \in \q{\alpha, \beta}$
        \item $\varphi(t) = x_0 + \int\limits_{t_0}^t f(s, \varphi(s))ds\ \forall t \in \q{\alpha, \beta}$
    \end{enumerate}
\end{defin}

\begin{lem}{}
    Функция $\varphi(t)$ -- решение задачи Коши тогда и только тогда, когда она является решением эквивалентного интегрального уравнения 
\end{lem}

\textit{Доказательство:}

\begin{itemize}
    \item[$\Rightarrow$] Пусть $\varphi(t)$ -- решение задачи Коши

        \begin{enumerate}
            \item $\varphi$ непрерывна -- очевидно 
            \item $(t, \varphi(t)) \in D\ \forall t \in \q{\alpha, \beta}$ -- то же условие
            \item $\varphi(t) = x_0 + \int\limits_{t_0}^t f(s, \varphi(s))ds\ \forall t \in \q{\alpha, \beta}$ -- получается интегрированием уравнения \\
            $\varphi'(t) = f(t, \varphi(t))$ с учетом начальных условий
        \end{enumerate}
    \item[$\Leftarrow$] Пусть $\varphi(t)$ -- решение интегрального уравнения
    
        \begin{enumerate}
            \item $\varphi$ непрерывна и есть интеграл от непрерывной функции -- значит дифференцируема
            \item $(t, \varphi(t)) \in D\ \forall t \in \q{\alpha, \beta}$ -- то же условие
            \item $\varphi'(t) = f(t, \varphi(t))\ \forall t \in \q{\alpha, \beta}$ -- получается дифференцированием интегрального уравнения
        \end{enumerate}
\end{itemize}

\begin{theo}{Теорема существования решений}
    Пусть правая часть $f(t, x)$ системы $x' = f(t, x)$ непрерывна в области $D \subset \R^{n + 1}$. Пусть $(t_0, x_0) \in D$. Тогда существует решение задачи Коши 

    $\begin{cases}
        x' = f(t, x) \\
        x(t_0) = x_0
    \end{cases}$

    определенное на промежутке $[t_0 - h, t_0 + h]$
\end{theo}

\begin{Remark}{}
    Этот промежуток называется промежутком Пеано
\end{Remark}

\textit{Доказательство:}

Будем вместо решения задачи Коши искать решение эквивалентного интегрального уравнения $x(t) = x_0 + \int\limits_{t_0}^t f(s, x(s))ds$

Поскольку $D$ -- область (открытое множество), выберем константы $a, b > 0$ такие, что $K := \{(t, x) : |t - t_0| \leq a;\ |x - x_0| \leq b\} \subset D$

$K$ -- компакт, значит непрерывная функция огр. Пусть $M = \max\limits_{(t, x) \in K} |f(t, x)|;\ h := \min(a, \frac{b}{M})$

\begin{Remark}{}
    Длина промежутка Пеано непрерывно зависит от начальной точки 
\end{Remark}

\begin{defin}{Векторные нормы}
    Понятие нормы в $\R^n$:

    \begin{enumerate}
        \item $\parl{x} \geq 0;\ \parl{x} = 0 \Leftrightarrow x = 0$
        \item $\parl{ax} = |a|\parl{x}\ \forall a \in \R, x \in \R^n$
        \item $\parl{x + y} \leq \parl{x} + \parl{y}\ \forall x, y \in \R^n$
    \end{enumerate}
\end{defin}

\begin{Example}{}
    \begin{enumerate}
        \item $\parl{x}_1 = |x| = \max(|x_1|, \ldots, |x_n|)$ -- с этой нормой и будем работать
        \item $\parl{x}_2 = \sqrt{x_1^2 + \ldots + x_n^2}$ -- евклидова норма
        \item $\parl{x}_3 = |x_1| + \ldots + |x_n|$
    \end{enumerate}
\end{Example}

\begin{defin}{Равностепенная непрерывность}
    Последовательность функций $\varphi_k : [\alpha, \beta] \to \R^n,\ k \in \N$ -- равностепенно непрерывна, если 
    
    $\forall \varepsilon > 0\ \exists \delta > 0 : \forall t_1, t_2 \in [\alpha, \beta],\ k \in \N$ верно $|t_1 - t_2| < \delta \Rightarrow |\varphi_k(t_1) - \varphi_k(t_2)| < \varepsilon$
\end{defin}

\begin{defin}{Равномерная ограниченность}
    Последовательность функций $\varphi_k : [\alpha, \beta] \to \R^n,\ k \in \N$ -- равномерно ограничена, если

    $\exists C > 0 : \forall t \in [\alpha, \beta],\ k \in \N$ верно $|\varphi_k(t)| \leq C$
\end{defin}

\begin{theo}{Теорема Арцела Асколи}
    Пусть последовательность функций $\varphi_k : [\alpha, \beta] \to \R^n,\ k \in \N$ равностепенно непрерывна и равномерно ограничена. Тогда существует равномерно сходящаяся подпоследовательность $\varphi_{n_k} \toto \varphi_*$ на $[\alpha, \beta]$
\end{theo}

\begin{defin}{Кусочно-гладкая функция}
    Функция $\varphi : [\alpha, \beta] \to \R^n$ называется кусочно-гладкой, если она непрерывна, имеет производную везде, кроме конечного числа точек, а в тех точках имеет односторонние пределы 
\end{defin}

\begin{defin}{$\varepsilon$-решение системы}
    Пусть $\varepsilon > 0$. Кусочно-гладкая функция $\varphi : [\alpha, \beta] \to \R^n$ называется $\varepsilon$-решением системы, если 

    \begin{enumerate}
        \item $(t, \varphi(t)) \in D\ \forall t \in [\alpha, \beta]$
        \item $|\varphi'(t) - f(t, \varphi(t))| \leq \varepsilon$ во всех точках, где производная определена
    \end{enumerate}
\end{defin}

\begin{lem}{}
    Пусть $\varepsilon_m \to 0$ и $\varphi_m(t)$ -- последовательность $\varepsilon_m$-решений системы на отрезке $[\alpha, \beta]$, такая, что $\varphi_m(t_0) = x_0;\ |f(t, \varphi_m(t))| \leq M$ и $\varphi_m \toto \varphi_*$. Тогда $\varphi_*$ -- решение задачи Коши
\end{lem}

\textit{Доказательство:}

Пусть $\Delta_m$ -- последовательность функций, заданных формулой 

$\varphi_m(t) = x_0 + \int\limits_{t_0}^t f(s, \varphi_m(s))ds + \Delta_m(t)$

Интегрируя неравенство $|\varphi'(t) - f(t, \varphi(t))| \leq \varepsilon_m$ от $t_0$ до $t$, с учетом того, что $\varphi_m(t_0) = \varphi(t_0) = x_0$, получаем $|\Delta_m(t)| \leq \varepsilon_m(\beta - \alpha)$

Переходя к пределу в первой формуле, получаем, что $\varphi_*(t)$ -- решение эквивалентного интегрального уравнения, а значит, и задачи Коши 

\begin{Remark}{}
    Далее, мы предложим метод построения таких приближенных решений. Мы будем строить эти решения на промежутке $[t_0, t_0 + h]$, построение на промежутке $[t_0 - h, t_0]$ аналогично
\end{Remark}

\begin{defin}{Ломаные Эйлера}
    Фиксируем $m \in \N$. Разделим отрезок $[t_0, t_0 + h]$ на $m$ равных частей: \\ $t_j = t_0 + \frac{hj}{m};\ j = 0, \ldots, m$

    Положим $\varphi_m(t_0) = x_0$ и последовательно определим $\varphi_m(t) = \varphi_m(t_j) + f(t_j, \varphi_m(t_j))(t - t_j)$ при $j = 0, \ldots, m - 1$ и $t \in [t_j, t_{j + 1}]$

    В частности $\varphi_m(t_{j + 1}) = \varphi_m(t_j) + f(t_j, \varphi_m(t_j))\frac{h}{m}$

    Если положить $A_j = (t_j, \varphi_m(t_j))$, то график $\varphi_m(t)$ -- ломаная, соединяющая точки $A_j$
\end{defin}

\begin{propos}{}
    $K := \{(t, x) := |t - t_0| \leq a;\ |x - x_0| \leq b\} \subset D$

    Для любого $m \in \N, t \in [t_0, t_0 + h]$ верно $(t, \varphi_m(t)) \in K$
\end{propos}

\textit{Доказательство:}

\begin{enumerate}
    \item $|t - t_0| \leq h = \min(a, \frac{b}{M})$
    \item $t^* = \min\limits_{t \in [t_0, t_0 + h]} \{|\varphi_m(t) - x_0| \geq b\}$
    
    С другой стороны, $|\varphi_m(t^*) - x_0| = |\varphi_m(t^*) - \varphi_m(t_0)| \leq \int\limits_t^{t^*} |\varphi_m'(s)|ds \leq M(t^* - t_0) \leq Mh \leq b$
\end{enumerate}

\begin{propos}{}
    $\forall \varepsilon > 0\ \exists m_0$, такое что при $m \geq m_0$ функция $\varphi_m$ является $\varepsilon$-решением системы 
\end{propos}

\textit{Доказательство:}

$|\varphi_m(t_1) - \varphi_m(t_2)| \leq M|t_1 - t_2|$

Если $t \in [t_j, t_{j + 1}]$, то $|t - t_j| \leq \frac{h}{m};\ |f(t_j, \varphi_m(t_j)) - f(t, \varphi_m(t))| \xrightarrow[m \to \infty]{} 0$ равномерно по $t$ 

\begin{propos}{}
    Функции $\varphi_m(t)$ равномерно ограничены 
\end{propos}

\textit{Доказательство:}

$|\varphi_m(t)| \leq M|t - t_0| + |x_0| \leq Mh + |x_0|$

\begin{propos}{}
    Функции $\varphi_m(t)$ равностепенно непрерывны
\end{propos}

\textit{Доказательство:}

$|\varphi_m(t_1) - \varphi_m(t_2)| \leq M|t_1 - t_2|$

\begin{theo}{Теорема Кнезера}
    В условиях теоремы существования, для любого $t_1 \in [t_0 - h, t_0 + h]$ множество значений решений задачи Коши $\{x(t_1) : x(t)\text{ -- решение}\}$ замкнуто и связно
\end{theo}

\begin{Exercise}{}
    Доказать замкнутость (пользуемся утверждениями 2.1-2.4, леммой 2.1 и теоремой Арцела-Асколи)
\end{Exercise}

\newpage 

\section{Лекция 3}

\begin{lem}{Лемма Гронуолла-Беллмана}
    Пусть $u(t) \geq 0;\ f(t) \geq 0; u(t), f(t) \in C[t_0, \infty)$, при этом \\
    для $t \geq t_0$ выполняется неравенство $u(t) \leq c + \int\limits_{t_0}^t f(t_1)u(t_1)dt_1$, где $c > 0$ -- константа

    Тогда при $t \geq t_0$ имеем оценку $u(t) \leq c \cdot \exp(\int\limits_{t_0}^t f(t_1)dt_1)$
\end{lem}

\textit{Доказательство:}

Из неравенства получаем $\frac{u(t)}{c + \int\limits_{t_0}^t f(t_1)u(t_1)dt_1} \leq 1$ и $\frac{f(t)u(t)}{c + \int\limits_{t_0}^t f(t_1)u(t_1)dt_1} \leq f(t)$

Т.к. $\frac{d}{dt}\left[ c + \int\limits_{t_0}^t f(t_1) u(t_1)dt_1 \right] = f(t)u(t)$, то проинтегрировав от $t_0$ до $t$, получим 

$\ln\left[ c + \int\limits_{t_0}^t f(t_1)u(t_1)dt_1 \right] - \ln c \leq \int\limits_{t_0}^t f(t_1)dt_1$, отсюда и из неравенства 

$u(t) \leq c + \int\limits_{t_0}^t f(t_1)u(t_1)dt_1 \leq c \cdot \exp(\int\limits_{t_0}^t f(t_1)dt_1)$, чтд

\begin{theo}{Следствие леммы Гронуолла-Беллмана}
    \begin{enumerate}
        \item $u(t) \leq \int\limits_{t_0}^t f(t_1)u(t_1)dt_1 \Rightarrow u(t) \equiv 0$
        \item $t \leq t_0$
        
        $u(t) \leq c + \left| \int\limits_{t_0}^t f(t_1)u(t_1)dt_1 \right| \Rightarrow u(t) \leq c \cdot \exp\left| \int\limits_t^{t_0} f(t_1)dt_1 \right|$
    \end{enumerate}
\end{theo}

\textit{Доказательство:}

В пункте 2 замена $s = -t$ 

\begin{lem}{Усиленная лемма Гронуолла-Беллмана}
    Пусть функция $u(x)$ неотрицательна и непрерывна в промежутке $[x_0, x_0 + h]$ и удовлетворяет там неравенству $0 \leq u(x) \leq A + B \int\limits_{x_0}^x u(t)dt + \varepsilon(x - x_0)$ при $A, B, \varepsilon \geq 0$

    Тогда при $x \in [x_0, x_0 + h]$ справедливо неравенство $u(x) \leq Ae^{B(x - x_0)} + \frac{\varepsilon}{B}(e^{B(x - x_0)} - 1)$
\end{lem}

\begin{Exercise}{}
    Доказать усиленную лемму Гронуолла-Беллмана
\end{Exercise}

\begin{defin}{Условие Липшица}
    Непрерывная функция (вектор-функция) $f : A \mapsto \R^n$ удовлетворяет условию Липшица, $f \in Lip(A)$, если существует такая константа $L > 0$, что $|f(x) - f(y) \leq L|x - y|$ для любых $x, y \in A$
\end{defin}

\begin{defin}{Локальное условие Липшица}
    Непрерывная функция (вектор-функция) $f : A \mapsto \R^n$ удовлетворяет локальному условию Липшица, если для любого $x_0 \in A$ существует окрестность $U$ точки $x_0$, в которой функция $f$ удовлетворяет условию Липшица
\end{defin}

\begin{lem}{}
    Функция $f : U \to \R^n$ -- непрерывно дифференцируема, где $U$ -- область в $\R^m$, значит $f$ удовлетворяет в этой области локальному условию Липшица
\end{lem}

\textit{Доказательство:}

Возьмем точку $x_0 \in U$ и замкнутый шарик $B$ с центром в $x_0$ такой, что $B \subset U$. Пусть $M = \max\limits_{x \in B}|Df(x)|$. Тогда по теореме о среднем $|f(x) - f(y)| \leq M|x - y|$ для любых $x, y \in B$

\begin{defin}{Условие Липшица по переменной $x$}
    Пусть $U \subset \R^{n + 1}_{t, x}$ -- область. Непрерывная вектор-функция $f : U \to \R^n$ удовлетворяет условию Липшица по переменной $x$, $f \in Lip_x(A)$ если существует такая константа $L > 0$, что $|f(t, x_1) - f(t, x_2)| \leq L|x_1 - x_2|$ для любых $(t, x_1), (t, x_2) \in U$
\end{defin}

\begin{defin}{Локальное условие Липшица по переменной $x$}
    Пусть $U \subset \R^{n + 1}_{t, x}$ -- область. Непрерывная вектор-функция $f : U \to \R^n$ удовлетворяет локальному условию Липшица по переменной $x$, $f \in Lip_{loc, x}(A)$, если для любой точки $(t_0, x_0) \in U$ существует окрестность $V$ этой точки и такая константа $L > 0$, что $|f(t, x_1) - f(t, x_2)| \leq L|x_1 - x_2|$ для любых $(t, x_1), (t, x_2) \in V$
\end{defin}

\begin{theo}{Теорема об условии Липшица в компакте}
    Пусть вектор-функция $f$ удовлетворяет локальному условию Липшица по $x$ в области $U$. Тогда для любого компакта $K \subset U$ эта функция липшицева по $x$ на этом компакте
\end{theo}

\textit{Доказательство:}

Пусть это утверждение неверно. Тогда существуют последовательности $(t_k, x_k) \in K$ и \\ 
$(t_k, y_k) \in K, x_k \neq y_k$, такие что $|f(t_k, x_k) - f(t_k, y_k)| \geq k|x_k - y_k|$

НУО можем считать, что $t_k \to t^*,\ x_k \to x^*,\ y_k \to y^*$. При этом $(t^*, x^*), (t^*, y^*) \in K$

Возможны два случая:

\begin{enumerate}
    \item $x^* \neq y^*$

    Тогда $\frac{|f(t_k, x_k) - f(t_k, y_k)|}{|x_k - y_k|} \to \infty,\ |x_k - y_k| \not\to 0 \Rightarrow |f(t_k, x_k) - f(t_k, y_k)|$ не ограничено. Противоречие (т.к. $f$ непрерывна на компакте)

    \item $x^* = y^*$
    
    В этом случае существует окрестность $U$ точки $(t^*, x^*)$ такая, что существует константа $L > 0$, что $|f(t, x) - f(t, y)| \leq L|x - y|$ для любых $(t, x), (t, y) \in U$

    Значит, такое неравенство выполнено для всех $(t_k, x_k), (t_k, y_k)$ начиная с некоторого номера. Противоречие
\end{enumerate}

\begin{theo}{Теорема единственности}
    Пусть $x' = f(t, x)$ -- система оду. $f : D \to \R^n;\ D \subset \R^{n + 1}$ -- область. Пусть $f$ непрерывна и локально липшицева по $x$ в области $D$

    Тогда для любой пары $(t_0, x_0) \in D$ задача Коши $\begin{cases}
        x' = f(t, x) \\
        x(t_0) = x_0
    \end{cases}$ имеет единственное решение
\end{theo}

Пусть утверждение теоремы неверно. Есть такая точка $(t_0, x_0) \in D$, что задача Коши имеет два различных решения $\varphi(t)$ и $\psi(t)$ на промежутке $[t_0 - h, t_0 + h]$

$K = \{(t, \varphi(t)) : t \in [t_0 - h, t_0 + h]\} \cup \{(t, \psi(t)) : t \in [t_0 - h, t_0 + h]\}$

Множество $K$ -- компакт. На нем выполнено глобальное условие Липшица по $x$, в частности 

$|f(t, \varphi(t)) - f(t, \psi(t))| \leq L|\varphi(t) - \psi(t)|$. Положим $u(t) = |\varphi(t) - \psi(t)|$

$\varphi(t) = x_0 + \int\limits_{t_0}^t f(s, \varphi(s))ds;\ \psi(t) = x_0 + \int\limits_{t_0}^t f(s, \psi(s))ds$

$\varphi(t) - \psi(t) = \int\limits_{t_0}^t \left[ f(s, \varphi(s)) - f(s, \psi(s)) \right] ds$

$|\varphi(t) - \psi(t)| = \left| \int\limits_{t_0}^t \left[ f(s, \varphi(s)) - f(s, \psi(s)) \right] ds \right| \leq \int\limits_{t_0}^t |f(s, \varphi(s)) - f(s, \psi(s))| ds \leq L\int\limits_{t_0}^t |\varphi(s) - \psi(s)| ds$

$u(t) \leq L\int\limits_{t_0}^t u(s) ds$ по следствию из леммы Гронуолла-Беллмана $u(t) \equiv 0$ и $\varphi(t) \equiv \psi(t)$

\begin{theo}{Следствие}
    Пусть $x' = f(t, x)$ -- система оду. $f : D \to \R^n;\ D \subset \R^{n + 1}$ -- область. Пусть $f$ непрерывна и непрерывно дифференцируема по $x$ в области $D$. Тогда для любой пары $(t_0, x_0) \in D$ задача Коши $\begin{cases}
        x' = f(t, x) \\
        x(t_0) = x_0
    \end{cases}$ имеет единственное решение
\end{theo}

\begin{Remark}{}
    Условие теоремы единственности достаточное, но не необходимое 

    $y' = y\ln|y|;\ y \neq 0;\ y' = 0$ при $y = 0$

    $y = 0$ или $\ln|\ln|y|| = x + c \Rightarrow y = e^{ce^x}$

    Единственность решений есть, а условия Липшица (даже локального) нет
\end{Remark}

\newpage 

\section{Лекция 4. Продолжение решений}

\begin{defin}{Продолжение решения}
    Пусть есть решения $\varphi : \q{a, b} \to \R^n,\ \psi : \q{a_1, b_1} \to \R^n$. Говорим, что решение $\psi$ есть продолжение решения $\varphi$ (продолжает решение $\varphi$), если 

    \begin{enumerate}
        \item $\q{a, b} \not\subseteq \q{a_1, b_1}$
        \item $\psi\mid_{\q{a, b}} = \varphi$
    \end{enumerate}
\end{defin}

\begin{defin}{Продолжимость влево}
    Решение $\varphi(t)$ называется продолжимым влево за $a$, если существует решение $\psi(t)$, продолжающее решение, и при этом $a_1 < a$
\end{defin}

\begin{defin}{Продолжимость вправо}
    Решение $\varphi(t)$ называется продолжимым вправо за $b$, если существует решение $\psi(t)$, продолжающее решение, и при этом $b_1 > b$
\end{defin}

\begin{defin}{Максимально продолженное решение}
    Решение $\varphi(t)$ называется непродолжимым или максимально продолженным, если решения $\psi(t)$, продолжающего $\varphi(t)$, не существует
\end{defin}

\begin{theo}{Теорема о продолжимости решений вправо за $b$}
    Пусть решение $\varphi(t)$ уравнения $x' = f(t, x)$ задано на промежутке $\langle a, b)$, причем существует предел $\lim\limits_{t \to b_-} \varphi(t) = x_0$ и $(b, x_0) \in D$. Тогда решение $\varphi(t)$ продолжимо вправо за $b$

    Теорема о продолжимости решения влево за $a$ выглядит аналогично
\end{theo}

\textit{Доказательство:}

Рассмотрим некоторое решение $\psi(t)$ задачи Коши для уравнения $x' = f(x, t)$ с начальными данными $x(b) = x_0$, заданное на промежутке Пеано $[b - h, b + h]$ и положим 

$\chi(t) = \left[ \begin{gathered}
    \varphi(t) \text{ если } t \in \langle a, b) \\
    \psi(t) \text{ если } t \in [b, b + h]
\end{gathered} \right.$

Достаточно показать, что $\chi(t)$ -- решение системы. Для этого достаточно проверить справедливость интегрального уравнения $\chi(t) = x_0 + \int\limits_b^t f(s, \chi(s))ds$

Пусть $\varphi(t_1) = x_1$, $\varphi(t) = x_1 + \int\limits_{t_1}^t f(s, \varphi(s))ds;\ \varphi(b) = x_0 = x_1 + \int\limits_{t_1}^b f(s, \varphi(s))ds$

\begin{defin}{Частничный порядок}
    Пусть $\mathfrak{M}$ -- некоторое множество. Отношение $\preccurlyeq$ на этом множестве называется частичным порядком, а само множество частично упорядоченным, если выполнены следующие соотношения

    \begin{enumerate}
        \item $a \preccurlyeq a$ для любого $a \in \mathfrak{M}$ (рефлексивность)
        \item $a \preccurlyeq b,\ b \preccurlyeq c \Rightarrow a \preccurlyeq c$ для любых $a, b, c \in \mathfrak{M}$ (транзитивность)
        \item $a \preccurlyeq b, b \preccurlyeq a \Rightarrow a = b$ (антисимметричность)
    \end{enumerate}
\end{defin}

\begin{Example}{}
    Обычный порядок на $\R$, делимость натуральных чисел, порядок по ключению для всех подмножеств некоторого множества $\mathfrak{A}$ ($A \preccurlyeq B \Leftrightarrow A \subset B$)
\end{Example}

\begin{defin}{Максимальный элемент}
    Элемент $a \in \mathfrak{M}$ называется максимальным, если $a \preccurlyeq b \Rightarrow b = a$
\end{defin}

\begin{defin}{Линейный порядок}
    Частично упорядоченное множество называется линейно упорядоченным (или цепью), если для любых $a, b \in \mathfrak{M}$ либо $a \preccurlyeq b$, либо $b \preccurlyeq a$
\end{defin}

\begin{defin}{Верхняя грань}
    Пусть $\mathfrak{A} \subset \mathfrak{M}$. Элемент $a \in \mathfrak{M}$ называется верхней гранью множества $\mathfrak{A}$, если $b \preccurlyeq a$ для любого $b \in \mathfrak{A}$
\end{defin}

\begin{lem}{Лемма Цорна}
    Если в частично-упорядоченном множестве $\mathfrak{M}$ каждое линейно упорядоченное подмножество имеет верхнюю грань, то само множество имеет максимальный элемент 
\end{lem}

\textit{Доказательство:}

Доказывается как следствие из аксиомы выбора 

\begin{theo}{}
    Для любого $(t_0, x_0) \in D$ существует максимально продолженное решение задачи коши $\begin{cases}
        x' = f(t, x) \\
        x(t_0) = x_0
    \end{cases}$
\end{theo}

\textit{Доказательство:}

Пусть $\mathfrak{M}$ -- множество всех решений задачи Коши. Для любых двух решений $\varphi, \psi$ задачи Коши говорим, что $\varphi \preccurlyeq \psi$, если $\psi$ продолжает $\varphi$ либо они совпадают. Для каждого решения $\varphi$ обозначим символом $I(\varphi$) область определения этого решения 

Пусть $\mathfrak{B}$ -- некоторое линейно упорядоченное множество решений задачи. Положим \\ $J = \bigcup\limits_{\varphi \in \mathfrak{B}} I(\varphi)$. Отметим, что для каждой точки $t \in J$ все значения $\varphi(t)$ при $\varphi \in \mathfrak{B}$ совпадают

В самом деле, если $\varphi, \psi \in \mathfrak{B}$, nо одно решение является продолжением другого и, если они определены в одной точке, то их значения там совпадают 

Тогда можно корректно определить решение $\eta(t)$, продолжающее все решения из $\mathfrak{B}$ (или совпадающее с кем-то из них). Это будет верхняя грань. Существование максимального элемента $\mathfrak{M}$ следует из леммы Цорна 

\begin{theo}{}
    Пусть $\varphi(t)$ -- максимально продолженное решение системы, заданное на отрезке $(a, b);\ K \subset D$ -- компакт. Тогда существует $\varepsilon > 0$, такое, что $\varphi(t) \notin K$ для любого $t \in (a, a + \varepsilon) \cup (b - \varepsilon, b)$
\end{theo}

\textit{Доказательство:}

Пусть не так. НУО существует $t_k \to b$ такая, что $\varphi(t_k) \in K$ для любого $k$. Можно считать, что $(t_k, \varphi(t_k)) \to (b, \varphi^*) \in K$

Тогда $(t, \varphi(t)) \to (b, \varphi^*)$ при $t \to b$. Существует $h_0 > 0, M > 0$ такие, что если $(t_0, \varphi(t_0)) \in K$, то решение $\varphi(t)$ определено на $[t_0 - h, t_0 + h]$ и на этом отрезке $|\varphi'(t)| \leq M$. Следует из построения промежутка Пеано

При больших $k\ t_{k + 1} - t_k < h;\ |\varphi(t) - \varphi(t_k)| \leq M(t - t_k)$ для любых $t \in [t_k, t_{k + 1}]$

Отсюда $(t, \varphi(t)) \to (b, \varphi^*)$ при $t \to b$. Тогда $\varphi(t)$ можно продолжить вправо за $b$. Противоречие

\begin{defin}{Почти линейные системы}
    Пусть $f : \R^{n + 1} \to \R^n$ -- непрерывная

    Система $x' = f(t, x)$ почти линейная, если существуют такие непрерывные функции $A(t), B(t) \geq 0$, что $|f(t, x) \leq A(t)|x| + B(t)$
\end{defin}

\begin{theo}{}
    Любое решение почти линейной системы продолжимо на $\R$
\end{theo}

\newpage

\section{Лекция 5. Линейные дифференциальные уравнения высших порядков}

\begin{defin}{Линейные дифференциальные уравнения высшего порядка}
    $a_n(x)y^{(n)} + a_{n - 1}(x)y^{(n - 1)} + \ldots + a_0(x)y = f(x)$ -- линейное неоднородное порядка $n$

    $a_n(x)y^{(n)} + a_{n - 1}(x)y^{(n - 1)} + \ldots + a_0(x)y = 0$ -- линейное однородное порядка $n$

    Предполагается, что все функции $a_i$ и $f$ непрерывны на некотором промежутке $I = (a, b)$, который может быть прямой или лучом. При этом $a_n(x) \neq 0$ для любого $x \in I$. Переменная $y$ может быть вещественной и комплексной
\end{defin}

\begin{theo}{Основные свойства решений линейного уравнения}
    Решения -- функции класса гладкости $C^n$, определенные на $I$ и дающие при подстановке в уравнение тождество. Уравнение сводится к системе $\begin{cases}
        y_1' = y_2 \\
        \ldots \\
        y_{n - 1}' = y_n \\
        y_n' = \frac{-a_{n - 1}(x)y_n - \ldots - a_0(x)y_1 + f(x)}{a_n(x)}
    \end{cases}$

    Отсюда следует, что решения уравнения с любыми начальными данными $x_0 \in I,\ (y_0, \ldots, y_n) \in \R^n$ определены, единственны и продолжимы на $I$
\end{theo}

\begin{defin}{Пространство решений однородной системы}
    Обозначим левую часть уравнения символом $\L := a_n(x)y^{(n)} + a_{n - 1}(x)y^{(n - 1)} + \ldots + a_0(x)y$

    Это линейный оператор: $\L(\alpha u + \beta v) = \alpha \L(u) + \beta \L(v)$

    Соответственно, множество решений однородного уравнения -- линейное пространство. Поскольку каждым начальным данным соответствует ровно одно решение, размерность пространства равна $n$

    Решения с начальными данными $(1, 0, \ldots, 0), (0, 1, 0, \ldots, 0), \ldots, (0, 0, \ldots, 1)$ образуют базис пространства решений. Это начальные данные $(y(x_0), y'(x_0), \ldots, y^{(n - 1)}(x_0))$
\end{defin}

\begin{Remark}{}
    Решения $\varphi_1(x), \ldots, \varphi_k(x)$ системы линейно зависмы, если существует константы $C_1, \ldots, C_k$, не все равные нулю, такие, что $C_1\varphi_1(x) + \ldots + C_n\varphi_n(x) \equiv 0$
\end{Remark}

\begin{defin}{Определитель Вронского}
    Пусть $\varphi_1(x), \ldots, \varphi_n(x)$ -- решения системы. Вронскианом или определителем Вронского этого семейства решений называется функция от $x$:

    $W(x) = W_{\varphi_1, \ldots, \varphi_n}(x) = \begin{vmatrix}
        \varphi_1(x) & \varphi_2(x) & \ldots & \varphi_n(x) \\
        \varphi_1'(x) & \varphi_2'(x) & \ldots & \varphi_n'(x) \\
        \vdots & \vdots & \ddots & \vdots \\
        \varphi_1^{(n - 1)}(x) & \varphi_2^{(n - 1)}(x) & \ldots & \varphi_n^{(n - 1)}(x)
    \end{vmatrix}$
\end{defin}

\begin{theo}{Свойства определителя Вронского}
    Пусть $\varphi_1(x), \ldots, \varphi_n(x)$ -- решения уравнения. Тогда равносильны следующие условия: 

    \begin{enumerate}
        \item $W_{\varphi_1, \ldots, \varphi_n}(x) \equiv 0$ 
        \item Существует $x_0 \in I$ такой, что $W_{\varphi_1, \ldots, \varphi_n}(x_0) = 0$
        \item Решения $\varphi_1(x), \ldots, \varphi_n(x)$ линейно зависимы 
    \end{enumerate}
\end{theo}

\textit{Доказательство:}

\begin{itemize}
    \item[$1 \Rightarrow 2$] Очевидно
    \item[$2 \Rightarrow 3$] $W_{\varphi_1, \ldots, \varphi_n}(x_0) = 0$. Тогда существуют константы $C_1, \ldots, C_n$, не все нулевые, такие что $\begin{cases}
        C_1\varphi_1(x_0) + \ldots + C_n\varphi_n(x_0) = 0 \\
        C_1\varphi_1'(x_0) + \ldots + C_n\varphi_n'(x_0) = 0 \\
        \ldots \\
        C_1\varphi_1^{(n - 1)}(x_0) + \ldots + C_n\varphi_n^{(n - 1)}(x_0) = 0
    \end{cases}$

    Положим $\psi(x) = C_1\varphi_1(x) + \ldots + C_n\varphi_n(x)$

    Тогда $\psi(x_0) = \psi'(x_0) = \ldots = \psi^{(n - 1)}(x_0) = 0 \Rightarrow \psi(x) \equiv 0$

    \item[$3 \Rightarrow 1$] $C_1\varphi_1(x) + \ldots + C_n\varphi_n(x) \equiv 0 \Rightarrow C_1\varphi_1^{(k)}(x) + \ldots + C_n\varphi_n^{(k)}(x) \equiv 0$ для любого $k = 1, \ldots, n - 1 \Rightarrow \\ \Rightarrow W_{\varphi_1, \ldots, \varphi_n}(x) \equiv 0$
\end{itemize}

\begin{theo}{Формула Остроградского-Лиувилля}
    Для любых $x_0, x \in I$ справедливо соотношение $W(x) = W(x_0) e^{-\int_{x_0}^x \frac{a_{n - 1}(t)}{a_n(t)}dt}$
\end{theo}

\textit{Доказательство:}

Для начала, докажем формулу подсчета производных определителей:

Пусть $U(x) = \begin{vmatrix}
    u_{11}(x) & \ldots & u_{1n}(x) \\
    u_{21}(x) & \ldots & u_{2n}(x) \\
    \vdots & \ddots & \vdots \\
    u_{n1}(x) & \ldots & u_{nn}(x)
\end{vmatrix}$ -- определитель, состоящий из дифференцируемых функций. Тогда 

$U'(x) = \begin{vmatrix}
    u_{11}'(x) & \ldots & u_{1n}'(x) \\
    u_{21}(x) & \ldots & u_{2n}(x) \\
    \vdots & \ddots & \vdots \\
    u_{n1}(x) & \ldots & u_{nn}(x)
\end{vmatrix} + \ldots + \begin{vmatrix}
    u_{11}(x) & \ldots & u_{1n}(x) \\
    u_{21}(x) & \ldots & u_{2n}(x) \\
    \vdots & \ddots & \vdots \\
    u_{n1}'(x) & \ldots & u_{nn}'(x)
\end{vmatrix}$

Формула следует из формулы производной произведения функций и из представления определителя в виде 

$U(x) = \begin{vmatrix}
    u_{11}(x) & \ldots & u_{1n}(x) \\
    u_{21}(x) & \ldots & u_{2n}(x) \\
    \vdots & \ddots & \vdots \\
    u_{n1}(x) & \ldots & u_{nn}(x)
\end{vmatrix} = \sum\limits_{p \in \prod_n} (-1)^{|p|} \prod\limits_{i = 1}^n u_{ip_i}(x)$

Здесь $\prod_n$ -- всевозможные перестановки $n$ элементов, $|p|$ -- четность перестановки

Отсюда $W'(x) = \begin{vmatrix}
    \varphi_1'(x) & \varphi_2'(x) & \ldots & \varphi_n'(x) \\
    \varphi_1'(x) & \varphi_2'(x) & \ldots & \varphi_n'(x) \\
    \vdots & \vdots & \ddots & \vdots \\
    \varphi_1^{(n - 1)}(x) & \varphi_2^{(n - 1)}(x) & \ldots & \varphi_n^{(n - 1)}(x)
\end{vmatrix} + \begin{vmatrix}
    \varphi_1(x) & \varphi_2(x) & \ldots & \varphi_n(x) \\
    \varphi_1''(x) & \varphi_2''(x) & \ldots & \varphi_n''(x) \\
    \vdots & \vdots & \ddots & \vdots \\
    \varphi_1^{(n - 1)}(x) & \varphi_2^{(n - 1)}(x) & \ldots & \varphi_n^{(n - 1)}(x)
\end{vmatrix} + \\ + \ldots + \begin{vmatrix}
    \varphi_1(x) & \varphi_2(x) & \ldots & \varphi_n(x) \\
    \vdots & \vdots & \ddots & \vdots \\
    \varphi_1^{(n - 1)}(x) & \varphi_2^{(n - 1)}(x) & \ldots & \varphi_n^{(n - 1)}(x) \\
    \varphi_1^{(n - 1)}(x) & \varphi_2^{(n - 1)}(x) & \ldots & \varphi_n^{(n - 1)}(x)
\end{vmatrix} + \begin{vmatrix}
    \varphi_1(x) & \varphi_2(x) & \ldots & \varphi_n(x) \\
    \varphi_1'(x) & \varphi_2'(x) & \ldots & \varphi_n'(x) \\
    \vdots & \vdots & \ddots & \vdots \\
    \varphi_1^{(n)}(x) & \varphi_2^{(n)}(x) & \ldots & \varphi_n^{(n)}(x)
\end{vmatrix} = \\ = 0 + 0 + \ldots + 0 + \begin{vmatrix}
    \varphi_1(x) & \varphi_2(x) & \ldots & \varphi_n(x) \\
    \varphi_1'(x) & \varphi_2'(x) & \ldots & \varphi_n'(x) \\
    \vdots & \vdots & \ddots & \vdots \\
    \varphi_1^{(n)}(x) & \varphi_2^{(n)}(x) & \ldots & \varphi_n^{(n)}(x)
\end{vmatrix} = \\ = \begin{vmatrix}
    \varphi_1(x) & \varphi_2(x) & \ldots & \varphi_n(x) \\
    \varphi_1'(x) & \varphi_2'(x) & \ldots & \varphi_n'(x) \\
    \vdots & \vdots & \ddots & \vdots \\
    -\frac{1}{a_n(x)} \sum\limits_{k = 0}^{n - 1}a_k(x)\varphi_1^{(k)}(x) & -\frac{1}{a_n(x)} \sum\limits_{k = 0}^{n - 1}a_k(x)\varphi_2^{(k)}(x) & \ldots & -\frac{1}{a_n(x)} \sum\limits_{k = 0}^{n - 1}a_k(x)\varphi_n^{(k)}(x)
\end{vmatrix} = \\ = -\sum\limits_{k = 0}^{n - 1} \frac{a_k(x)}{a_n(x)} \begin{vmatrix}
    \varphi_1(x) & \varphi_2(x) & \ldots & \varphi_n(x) \\
    \varphi_1'(x) & \varphi_2'(x) & \ldots & \varphi_n'(x) \\
    \vdots & \vdots & \ddots & \vdots \\
    \varphi_1^{(k)}(x) & \varphi_2^{(k)}(x) & \ldots & \varphi_n^{(k)}(x) 
\end{vmatrix} = -\frac{a_{n - 1}(x)}{a_n(x)} \begin{vmatrix}
    \varphi_1(x) & \varphi_2(x) & \ldots & \varphi_n(x) \\
    \varphi_1'(x) & \varphi_2'(x) & \ldots & \varphi_n'(x) \\
    \vdots & \vdots & \ddots & \vdots \\
    \varphi_1^{(n - 1)}(x) & \varphi_2^{(n - 1)}(x) & \ldots & \varphi_n^{(n - 1)}(x)
\end{vmatrix} = \\ = -\frac{a_{n - 1}(x)}{a_n(x)} W(x) = W'(x)$

\begin{Remark}{Зачем нужна формула Остроградского-Лиувилля?}
    $y'' + xy' - y = 0$

    Догадались, что решение $y = x$. Пусть другое решение -- $\varphi(x)$

    $\begin{vmatrix}
        \varphi(x) & x \\
        \varphi'(x) & 1
    \end{vmatrix} = W(0) e^{-\frac{x^2}{2}}$. НУО $W(0) = -1$

    $y'x - y = e^{-\frac{x^2}{2}}$. $\varphi(x)$ -- решение. Решение линейного однородного уравнения $y'x - y = 0$ -- это $y = Cx$

    $y = ux \Rightarrow u'x^2 + ux - ux = e^{-\frac{x^2}{2}} \Rightarrow u' = x^{-2} e^{-\frac{x^2}{2}}$
\end{Remark}

\begin{Exercise}{}
    Каков должен быть минимальный порядок линейного однородного уравнения, определенного на всей оси, чтобы у него были одновременно решения $y = x$ и $y = \sin x$
\end{Exercise}

\newpage 

\section{Лекция 6. Линейные дифференциальные уравнения с постоянными коэффициентами}

\begin{defin}{Квазимногочлены}
    Так называются выражения $f(x) = P_1(x)e^{\lambda_1x} + \ldots + P_n(x)e^{\lambda_nx}$, где $\lambda_1, \ldots, \lambda_n \in \C$ и $P_1(x), \ldots, P_n(x)$ -- многочлены с комплексными коэффициентами
\end{defin}

\begin{Example}{}
    \begin{enumerate}
        \item $x\sin x$
        \item $e^x + x^2 + e^{2x}$
    \end{enumerate}
\end{Example}

\begin{lem}{}
    Пусть наборы $(k_1, \lambda_1), \ldots, (k_m, \lambda_m);\ k_j \in \N \cup \{0\};\ \lambda_j \in \C$ попарно различны (то есть для любых двух пар либо первая, либо вторая компоненты разные)
    
    Тогда функции $\{x^{k_j}e^{\lambda_jx},\ j = 1, \ldots, m\}$ линейно независимы
\end{lem}

\begin{lem}{}
    Для любого квазимногочлена $f(x) = P_1(x)e^{\lambda_1x} + \ldots + P_n(x)e^{\lambda_nx}$ верно следующее

    $f(x) \equiv 0 \Leftarrow P_1(x) \equiv P_2(x) \equiv \ldots \equiv P_n(x) \equiv 0$
\end{lem}

\textit{Доказательство:}

Отметим следующий факт. Пусть $\lambda \neq 0,\ k \in \N \cup \{0\}$. Тогда $(x^ke^{\lambda x})' = (\lambda x^k + kx^{k - 1})e^{\lambda x}$, то есть справа при $e^{\lambda x}$ стоит многочлен степени $k$. Отсюда, следует, что если $\lambda \neq 0$ и $P(x)$ -- многочлен, то $(P(x)e^{\lambda x})' = Q(x)e^{\lambda x}$, где $Q(x)$ -- многочлен и $\deg Q = \deg P$

Будем доказывать индукцией по $n$ -- числу слагаемых 

\textbf{База:} $n = 1$. Очевидно. $P(x)e^{\lambda x} \equiv 0 \Rightarrow P(x) \equiv 0$

\textbf{Переход:} $n \to n + 1$. Пусть $P_1(x)e^{\lambda_1x} + \ldots + P_ne^{\lambda_nx} + P_{n + 1}(x)e^{\lambda_{n + 1}x} \equiv 0$, причем $\deg P_{n + 1} = m$

$P_1(x)e^{(\lambda_1 - \lambda_{n + 1})x} + \ldots + P_n(x)e^{(\lambda_n - \lambda_{n + 1})x} + P_{n + 1}(x) \equiv 0$

Продифференцируем полученное равенство $m + 1$ раз по $x$. Получим 

$Q_1(x)e^{(\lambda_1 - \lambda_{n + 1})x} + \ldots + Q_n(x)e^{(\lambda_n - \lambda_{n + 1})x} \equiv 0$, причем $\deg P_j = \deg Q_j$

Степень нулевого многочлена -- $- \infty$. В силу индукционного предположения получаем, что $P_1(x) \equiv \ldots \equiv P_n(x) \equiv 0$. Тогда в силу базы $P_{n + 1}(x) \equiv 0$

\begin{nota}{}
    Сейчас будем рассматривать уравнения с постоянными коэффициентами: 

    $\L y := a_ny^{(n)} + a_{n - 1}y^{(n - 1)} + \ldots + a_1y' + a_0y = 0$, где $a_0, \ldots, a_n \in \R$ и $a_n \neq 0$
\end{nota}

\begin{Remark}{}
    Почти все результаты лекции применимы к случаю уравнений с комплексными коэффициентами 
\end{Remark}

\begin{defin}{Характеристический многочлен}
    Многочлен $\chi(\lambda) := a_n\lambda^n + a_{n - 1}\lambda^{n - 1} + \ldots + a_1\lambda + a_0$ называется характеристическим многочленом уравнения $\L y = 0$
\end{defin}

\begin{Example}{}
    $y'' + 3y' + 2y = 0$

    $\lambda^2 + 3\lambda + 2 = \chi(\lambda)$ -- характеристическое уравнение 

    Заметим, что для любого $\lambda \in \C, m \in \N, (e^{\lambda x})^{(m)} = \lambda^me^{\lambda x}$
    
    Отсюда следует, что $\L[e^{\lambda x}] = \chi(\lambda)e^{\lambda x}$

    В частсности, если $\lambda$ -- корень характеристического многочлена, то $e^{\lambda x}$ -- решение уравнения. Например, $e^{-x}$ и $e^{-2x}$ -- решения данного уравнения
\end{Example}

\begin{theo}{}
    Пусть $\lambda_0$ -- корень характеристического уравнения кратности $k \in \N$. Тогда функции $e^{\lambda_0x}, xe^{\lambda_0x}, \ldots, x^{k - 1}e^{\lambda_0x}$ -- решения уравнения $\L y = 0$ 
\end{theo}

\textit{Доказательство:}

Если $\lambda_0$ -- корень кратности $k$, то $\chi(\lambda_0) = \chi'(\lambda_0) = \ldots = \chi^{(k - 1)}(\lambda_0) = 0$

С другой стороны, пусть $m \in \{0, \ldots, k - 1\}$. Продифференцируем $\L[e^{\lambda x}] = \chi(\lambda)e^{\lambda x}$ $m$ раз по $\lambda$. С одной стороны, дифференцирования по $x$ и $\lambda$ коммутируют между собой и с умножением на константы. Следовательно $\grad{^m}{\lambda^m}\L[e^{\lambda x}] = \L[\grad{^m}{\lambda^m}e^{\lambda x}] = \L[x^me^{\lambda x}]$

С другой стороны, по формуле производной произведения $\grad{^m}{\lambda^m}(\chi(\lambda)e^{\lambda x}) = \sum\limits_{j = 0}^m C_m^j\chi^{(j)}(\lambda) x^{m - j}e^{\lambda x}$

Подставляя $\lambda = \lambda_0$, получаем, что $\L[x^me^{\lambda_0x}] = 0$, значит $x^me^{\lambda_0x}$ -- решение уравнения

\begin{Example}{}
    \begin{enumerate}
        \item $y''' + 5y'' + 6y' = 0$
        
        $\chi(\lambda) = \lambda^3 + 5\lambda^2 + 6\lambda$. Корни -- $0, -2, -3$. Решения: $1, e^{-2x}, e^{-3x}$

        \item $y''' + y'' - y' - y = 0$
        
        $\chi(\lambda) = \lambda^3 + \lambda^2 - \lambda - 1$. Корни -- $-1$ (кратности 2), $1$. Решения: $e^{-x}, xe^{-x}, e^x$
    \end{enumerate}
\end{Example}

\begin{lem}{}
    Пусть $\varphi(x)$ -- комплексное решение уравнения с вещественными коэффициентами, тогда $\re \varphi(x)$ и $\im \varphi(x)$ -- тоже решения 
\end{lem}

\begin{Example}{}
    $y'' - 8y' + 25y = 0 \Rightarrow \lambda^2 - 8\lambda + 25 = 0 \Rightarrow \lambda_{1, 2} = 4 \pm 3i$

    Будут решения $e^{(4 \pm 3i)x} = e^{4x}(\cos 3x \pm i\sin 3x)$ 
\end{Example}

\begin{Remark}{}
    Если $\lambda = \alpha + i\beta$ -- корень характеристического уравнения кратности $k$, то $\ol{\lambda} = \alpha - i\beta$ -- тоже. Этой паре корней отвечает набор из $2k$ решений:

    $e^{\alpha x}\cos \beta x, xe^{\alpha x}\cos \beta x, \ldots, x^{k - 1}e^{\alpha x}\cos \beta x$

    $e^{\alpha x}\sin \beta x, xe^{\alpha x}\sin \beta x, \ldots, x^{k - 1}e^{\alpha x}\sin \beta x$
\end{Remark}

\begin{Example}{}
    $y'''' + 2y'' + y = 0 \Rightarrow \lambda^4 + 2\lambda^2 + 1 = 0 \Rightarrow \lambda = \pm i$ (кратности 2)

    Решения: $\cos x, x\cos x, \sin x, x\sin x$
\end{Example}

\begin{defin}{Уравнения Эйлера}
    Это уравнения вида $a_nx^ny^{(n)} + a_{n - 1}x^{n - 1}y^{(n - 1)} + \ldots + a_1xy' + a_0y = 0$

    Они заданы при $x > 0$ и при $x < 0$. Соответственно, замены $x = e^t$ и $x = -e^t$ ($t = \ln|x|$) сводят уравнения Эйлера к уравнению с постоянными коэффициентами 

    Характеристический многочлен имеет вид 
    
    $\chi(\lambda) = a_n\lambda(\lambda - 1)\ldots(\lambda - n + 1) + a_{n - 1}\lambda(\lambda - 1)\ldots(\lambda - n + 2) + \ldots + a_1\lambda + a_0$
\end{defin}

\begin{lem}{}
    Пусть $\lambda_0$ -- корень характеристического уравнения кратности $k \in \N$. Тогда ему отвечает набор решений $|x|^{\lambda_0}, |x|^{\lambda_0}\ln|x|, \ldots, |x|^{\lambda_0}(\ln|x|)^{k - 1}$

    Если есть пара комплексно сопряженных корней $\alpha \pm i\beta$ кратности $k$, то ей отвечают решения 

    $|x|^{\alpha}\cos(\beta \ln|x|), |x|^{\alpha}\ln|x|\cos(\beta \ln|x|), \ldots, |x|^{\alpha}(\ln|x|)^{k - 1}\cos(\beta \ln|x|)$

    $|x|^{\alpha}\sin(\beta \ln|x|), |x|^{\alpha}\ln|x|\sin(\beta \ln|x|), \ldots, |x|^{\alpha}(\ln|x|)^{k - 1}\sin(\beta \ln|x|)$
\end{lem}

\begin{Example}{}
    $x^2y'' + xy' - y = 0 \Rightarrow \chi(\lambda) = \lambda(\lambda - 1) + \lambda - 1 = \lambda^2 - 1 \Rightarrow \lambda = \pm 1$
\end{Example}

\begin{defin}{Почти линейная система}
    Пусть $f : \R^{n + 1} \to \R^n$ -- непрерывная функция

    Система $x' = f(t, x)$ почти линейная, если существуют такие непрерывные на всей оси функции $A(t), B(t) \geq 0$, что $|f(t, x)| \leq A(t)|x| + B(t)$
\end{defin}

\begin{theo}{}
    Любое решение почти линейной системы продолжимо на $\R$
\end{theo}

\textit{Доказательство:}

Пусть $\parl{x}$ -- евклидова норма. Ясно, что для любого решения $x(t)$ системы имеем 

$\frac{d}{dt}(\parl{x(t)}^2) = \frac{d}{dt}\q{x(t), x(t)} = 2\q{x(t), x'(t)}$

Пусть максимальный промежуток существования решения $x(t)$ -- интервал $(a, b)$, содержащий точку $t_0$. Из последнего неравенства следует, что 

$\parl{x(t)}^2 \leq \parl{x(t_0)}^2 + \left|\int\limits_{t_0}^t \left|\frac{d}{ds}\parl{x(s)}^2\right|\right| = \parl{x(t_0)}^2 + 2\left|\int\limits_{t_0}^t |\q{x(s), x'(s)}| ds\right| \leq \\ \leq \parl{x(t_0)}^2 + 2\left|\int\limits_{t_0}^t |\q{x(s), f(s, x(s))}|ds\right| \leq \parl{x(t_0)}^2 + 2\left|\int\limits_{t_0}^t |x(s)||f(s, x(s))||ds \right| \leq \\ \leq \parl{x(t_0)}^2 + 2\left|\int\limits_{t_0}^t ||x(s)||(A(s)||x(s)|| + B(s))ds \right| \leq (*)$

Пусть решение не продолжимо вправо за $b$. Тогда $b < + \infty$ и решение $x(t)$ стремится к бесконечности при $t \to b$ (оно же обязано покидать любой компакт!)

Тогда положим $C_1 = 2\max\limits_{t \in [t_0, b]} (\max(A(t)), B(t)), C_2 = C_1|b - t_0|$

Воспользовавшись неравенством $\parl{x(s)} \leq \frac{1}{2}(\parl{x(s)}^2 + 1)$, для оценки $B(s)\parl{x(s)}$, получаем 

$(*) \leq \parl{x(t_0)}^2 + C_1\left|\int\limits_{t_0}^t \parl{x(s)}^2ds\right| + C_1|t - t_0|$

Отсюда и из леммы Гронуолла следует, что $\parl{x(t)}^2$ ограничено на $[t_0, b]$. Противоречие 

Продолжимость влево на $(- \infty, t_0]$ проверяется аналогично

\newpage 

\section{Лекция 7. Линейные неоднородные уравнения}

\begin{Example}{Пример дифференциальных уравнений: пружинный маятник}
    Пусть есть грузик массы $m$ на пружине жесткости $k$, испытывающий при движении силу трения, пропорциональную скорости
    
    Тогда по второму закону Ньютона $ma = F_{\text{упр}} + F_{\text{тр}} + F(t)$

    Это можно записать в виде $mx'' + bx' + kx = F(t)$. При $F(t) = 0$ (свободные колебания)

    \begin{enumerate}
        \item $b^2 < 4km, \omega^2 = \frac{b^2}{4} - km \Rightarrow x = C_1e^{-\frac{bt}{2}}\cos(\omega t) + C_2e^{-\frac{bt}{2}}\sin(\omega t)$ 
        \item $b^2 = 4km \Rightarrow x = C_1e^{-\frac{bt}{2}} + C_2te^{-\frac{bt}{2}}$
        \item $b^2 > 4km \Rightarrow x = C_1e^{-a_1 t} + C_2e^{-a_2 t}$, где $a_{1, 2} = -\frac{b}{2} \pm \sqrt{km - \frac{b^2}{4}}$
    \end{enumerate}
\end{Example}

\begin{Example}{Пример дифференциальных уравнений: колебательный контур в электрической цепи}
    Пусть есть электрический колебательный контур, содержащий 

    \begin{enumerate}
        \item Источник переменного тока $V$
        \item Резистор сопротивления $R$
        \item Катушку индуктивности $L$
        \item Конденсатор емкости $C$
    \end{enumerate}

    Такие цепи используются, например, в радиоприемниках и телефонах. Уравнение свободных колебаний имеет вид $\frac{d^2}{dt^2}I(t) + \frac{R}{L}\frac{d}{dt}I(t) + \frac{1}{LC}I(t) = 0$

    Или, используя обозначения $\alpha = \frac{R}{2L}$ и $\omega_0 = \frac{1}{\sqrt{LC}}$: $\frac{d^2}{dt^2}I(t) + 2\alpha\frac{d}{dt}I(t) + \omega_0^2I(t) = 0$

    Интенсивность колебаний неоднородной системы максимальна, когда период колебаний $V$ близок к $\frac{2\pi}{\omega_0}$ 
\end{Example}

\begin{defin}{Линейные неоднородные уравнения}
    Так называются уравнения вида $a_n(x)y^{(n)} + a_{n - 1}(x)y^{(n - 1)} + \ldots + a_1(x)y' + a_0(x)y = f(x)$, или, если обозначить $\L y := a_n(x)y^{(n)} + a_{n - 1}(x)y^{(n - 1)} + \ldots + a_1(x)y' + a_0(x)y$, то $\L y = f(x)$

    Для решения уравнения линейного неоднородного уравнения сначала надо решить линейное однородное уравнение вида $\L y = 0$ 
\end{defin}

\begin{lem}{}
    Пусть $y^*(x)$ -- некоторое решение линейного неоднородного уравнения. Тогда общее решение уравнения имеет вид $y_\text{ОН}(x) = C_1\varphi_1(x) + \ldots + C_n\varphi_n(x) + y^*(x)$
\end{lem}

\textit{Доказательство:}

Достаточно проверить, что разница двух решений неоднородно уравнения -- решение однородного уравнения

$\L y_1 = \L y_2 = f \Rightarrow \L(y_1 - y_2) = 0$ 

\begin{Example}{}
    $y'' + y = e^x$. Решение однородного $y'' + y = 0$ -- $y_\text{ОО} = C_1\sin x + C_2\cos x$

    Решение неоднородного $y = \frac{e^x}{2} \Rightarrow y = C_1\sin x + C_2\cos x + \frac{e^x}{2}$
\end{Example}

\begin{lem}{Принцип суперпозиции}
    Пусть $\L y_1 = f_1(x), \L y_2 = f_2(x)$, $a, b$ -- константы. Тогда $\L(ay_1 + by_2) = af_1(x) + bf_2(x)$
\end{lem}

\textit{Доказательство:}

Следует из линейности оператора $\L$

\begin{theo}{}
    Пусть $y_\text{ОО}(x) = C_1\varphi_1(x) + \ldots + C_n\varphi_n(x)$ -- общее решение уравнения. Тогда функция $y^*(x) = C_1(x)\varphi_1(x) + \ldots + C_n(x)\varphi_n(x)$ будет решением уравнения, где функции $C_1(x), \ldots, C_n(x)$ -- первообразные (любые) решений следующей алгебраической системы 

    $\begin{cases}
        C_1'(x)\varphi_1(x) + C_2'(x)\varphi_2(x) + \ldots + C_n'(x)\varphi_n(x) = 0 \\
        C_1'(x)\varphi_1'(x) + C_2'(x)\varphi_2'(x) + \ldots + C_n'(x)\varphi_n'(x) = 0 \\
        \ldots \\
        C_1'(x)\varphi_1^{(n - 2)}(x) + C_2'(x)\varphi_2^{(n - 2)}(x) + \ldots + C_n'(x)\varphi_n^{(n - 2)}(x) = 0 \\
        C_1'(x)\varphi_1^{(n - 1)}(x) + C_2'(x)\varphi_2^{(n - 1)}(x) + \ldots + C_n'(x)\varphi_n^{(n - 1)}(x) = \frac{f(x)}{a_n(x)}
    \end{cases}$
\end{theo}

\begin{Remark}{}
    Определитель системы -- вронскиан системы решений $\varphi_1, \ldots, \varphi_n$. Следовательно решение алгебраической системы всегда существуют и единственно 
\end{Remark}

\textit{Доказательство:}

Продифференцируем формулу $y^*(x) = C_1(x)\varphi_1(x) + \ldots + C_n(x)\varphi_n(x)$

Получим $y^{*'}(x) = C_1'(x)\varphi_1(x) + \ldots + C_n'(x)\varphi_n(x) + C_1(x)\varphi_1'(x) + \ldots + C_n(x)\varphi_n'(x) = \\ = C_1(x)\varphi_1'(x) + \ldots + C_n(x)\varphi_n'(x)$, по первому равенству из системы

$y^{*''}(x) = C_1\varphi_1''(x) + \ldots + C_n\varphi_n''(x)$, по второму равенству из системы итд 

$y^{*(n)}(x) = C_1'(x)\varphi_1^{(n - 1)}(x) + \ldots + C_n'(x)\varphi_n^{(n - 1)}(x) + C_1(x)\varphi_1^{(n)}(x) + \ldots + C_n(x)\varphi_n^{(n)}(x) = \\ = C_1(x)\varphi_1^{(n)}(x) + \ldots + C_n(x)\varphi_n^{(n)}(x) + \frac{f(x)}{a_n(x)}$ и подставляем все это в уравнение 

$\L y^* = \sum\limits_{j = 0}^n a_j(x)y^{*(j)}(x) = a_n(x) \frac{f(x)}{a_n(x)} + \sum\limits_{j = 0}^n a_j(x) \sum\limits_{k = 1}^n C_k(x)\varphi_k^{(j)}(x) = f(x) + \sum\limits_{k = 1}^n C_k(x) \sum\limits_{j = 0}^n a_j(x)\varphi_k^{(j)}(x) = \\ = f(x) + \sum\limits_{k = 1}^n C_k(x) \L \varphi_k(x) = f(x)$

\begin{Remark}{}
    Если мы возьмем в качестве функций $C_j(x)$ какие-то другие первообразные решения нашего уравнения, то к решению $y^*(x)$ добавится какое-то решение линейного однородного уравнения. Вид общего решения неоднородного уравнения не изменится
\end{Remark}

\begin{Example}{}
    $y'' + 2y' + y = -\frac{e^{-x}}{x^2}$

    Решение линейного однородного уравнения $\lambda^2 + 2\lambda + 1 = 0 \Rightarrow \lambda = -1$ (кратности 2)

    $y_\text{ОО} = C_1e^{-x} + C_2xe^{-x}$

    $y^*(x) = C_1(x)e^{-x} + C_2(x)xe^{-x}$

    $\begin{cases}
        C_1'(x)e^{-x} + C_2'(x)xe^{-x} = 0 \\
        -C_1'(x)e^{-x} + C_2'(x)(1 - x)e^{-x} = -\frac{e^{-x}}{x^2}
    \end{cases} \Rightarrow \begin{cases}
        C_1' + C_2'x = 0 \\
        C_2' = -\frac{1}{x^2}
    \end{cases} \Rightarrow \begin{cases}
        C_1' = \frac{1}{x} \\
        C_2' = -\frac{1}{x^2}
    \end{cases}$

    Можно взять $C_1(x) = \ln|x|, C_2(x) = \frac{1}{x}$. Тогда получим $y^*(x) = \ln|x|e^{-x} + e^{-x}$

    Замечаем, что $\ln|x|e^{-x}$ -- тоже реение уравнения

    \textbf{Ответ:} $y = C_1e^{-x} + C_2xe^{-x} + \ln|x|e^{-x}$
\end{Example}

\begin{defin}{Метод неопределенных коэффициентов}
    Пусть все коэффициенты $a_j(x)$ постоянны, а функция $f(x)$ в правой части уравнения имеет вид квазимногочлена, то есть уравнение имеет вид 

    $a_ny^{(n)} + \ldots + a_1y' + a_0y = P(x)e^{\lambda_0 x}$

    В правой части может стоять линейная комбинация квазимногочленов, тогда мы пользуемся принципом суперпозиции

    В частности, $\lambda_0$ может быть комплексным числом, то есть можем работать с экспонентами, умноженными на синусы и косинусы
\end{defin}

\begin{defin}{Резонанс и нерезонанс}
    Пусть $\chi(x) = a_n\lambda^n + \ldots + a_1\lambda + a_0$ -- характеристический многочлен, соответствующего линейного однородного уравнения с постоянными коэффициентами 

    Тогда для уравнения возможны два случая: 

    \begin{enumerate}
        \item $\chi(\lambda_0) \neq 0$ -- нерезонансный случай
        \item $\chi(\lambda_0) = 0$ -- резонансный случай. Пусть $k \in \{1, \ldots, n\}$ таково, что 
        
        $\chi(\lambda_0) = \ldots = \chi^{(k - 1)}(\lambda_0) = 0$, но $\chi^{(k)}(\lambda_0) \neq 0$. Тогда говорим о резонансе порядка $k$
    \end{enumerate}

    Отсутствие резонанса -- это резонанс порядка $0$
\end{defin}

\begin{theo}{}
    Пусть в уравнении $a_ny^{(n)} + \ldots + a_1y' + a_0y = P(x)e^{\lambda_0 x}$ $\deg P(x) = m$. Тогда 

    \begin{enumerate}
        \item Если $\chi(\lambda_0) \neq 0$, то есть имеет место нерезонанс, то указанное уравнение имеет решение вида $y^*(x) = Q(x)e^{\lambda_0 x}$, где $Q(x)$ -- некоторый многочлен той же степени, что и $P(x)$
        \item Если имеет место резонанс порядка $k$, то указанное уравнение имеет решение вида $y^*(x) = x^kQ(x)e^{\lambda_0 x}$, где $\deg Q = \deg P$
    \end{enumerate}
\end{theo}

\textit{Доказательство:}

Достаточно доказать случай 2, случай 1 доказывается так же ($k = 0$)

Пусть $P(x) = \sum\limits_{l = 0}^m p_lx^l$. Положим $Q(x) = \sum\limits_{j = 0}^m q_jx^j$. 

Подставляем функцию $y^*(x) = x^kQ(x)e^{\lambda_0 x} = \sum\limits_{j = 0}^m q_jx^{k + j}e^{\lambda_0 x}$ в уравнение

В левой части этого уравнения получаем 

$\L y^* = \sum\limits_{j = 0}^m q_j\L[x^{k + j} e^{\lambda_0 x}] = \sum\limits_{j = 0}^m q_j \grad{^{k +j}}{\lambda^{k + j}} \L[e^{\lambda x}] \Big|_{\lambda = \lambda_0} = \sum\limits_{j = 0}^m q_j \grad{^{k + j}}{\lambda^{k + j}} (\chi(\lambda)e^{\lambda x}) \Big|_{\lambda = \lambda_0} x^l = \\ = \sum\limits_{j = 0}^m q_j \sum\limits_{l = 0}^{k + j} C_{k + j}^l x^k \chi^{(k + j - l)}(\lambda_0)e^{\lambda_0 x} = \sum\limits_{j = 0}^m q_j \sum\limits_{l = 0}^j C_{k + j}^l x^l \chi^{(k + j - l)}(\lambda_0)e^{\lambda_0 x} = \\ = \sum\limits_{l = 0}^m \left( \sum\limits_{j = l}^m q_j C_{k + j}^l \chi^{(k + j - l)}(\lambda_0) \right) x^l e^{\lambda_0 x}$

Это выражение должно равняться $f(x) = \sum\limits_{l = 0}^m p_l x^l e^{\lambda_0 x}$

Таким образом, для коэффициентов $q_j$ получаем уравнения 

При $x^m : C_{m + k}^m \chi^{(k)}(\lambda_0) q_m = p_m$

При $x^{m - 1} : C_{m - 1 + k}^{m - 1} \chi^{(k)}(\lambda_0) q_{m - 1} + C_{m + k}^{m - 1} \chi^{(k + 1)}(\lambda_0) q_m = p_{m - 1}$

$\ldots$

При $1 : \chi^{(k)}(\lambda_0) q_0 + \chi^{(k + 1)}(\lambda_0) q_1 + \ldots + \ldots + \chi^{(k + m)}(\lambda_0) q_m = p_0$

Если коэффициенты $p_0, \ldots, p_m$ известны, то коэффициенты $q_0, \ldots, q_m$ могут быть однозначно найдены последовательно, начиная с $q_m$

\begin{Example}{}
    $y'' - 3y' + 2y = (x^2 + 1)e^{3x} + e^x$

    \begin{enumerate}
        \item $y'' - 3y' + 2y = 0 \Rightarrow y_\text{ОО} = C_1e^{2x} + C_2e^{x}$
        
        \item $f(x) = f_1(x) + f_2(x) = (x^2 + 1)e^{3x} + e^x$

        Рассмотрим два уравнения 

        \begin{itemize}
            \item $y'' - 3y' + 2y = (x^2 + 1)e^{3x}$
            \item $y'' - 3y' + 2y = e^x$
        \end{itemize}

        \item Первое уравнение нерезонансно $3 \notin \{1, 2\}$. Имеет решение вида 
        
        $y_1^* = (ax^2 + bx + c)e^{3x}$
        
        $y_1^{*'} = (3ax^2 + (2a + 3b)x + (b + 3c))e^{3x}$

        $y_1^{*''} = (9ax^2 + (8a + 9b)x + (2a + 4b + 9c))e^{3x}$

        $\L y_1^* = (2ax^2 + (2a + 2b)x + (2a + b + 2c))e^{3x}$

        $\begin{cases}
            2a = 1 \\
            2a + 2b = 0 \\
            2a + b + 2c = 1
        \end{cases} \Rightarrow \begin{cases}
            a = \frac{1}{2} \\
            b = -\frac{1}{2} \\
            c = \frac{1}{4}
        \end{cases}$

        $y_1^* = \left(\frac{1}{2}x^2 - \frac{1}{2}x + \frac{1}{4}\right)e^{3x}$

        Второе уравнение резонансно $\lambda_1 = 1$, резонанс порядка 1

        $y_2^* = Axe^x,\ y_2^{*'} = A(x + 1)e^x,\ y_2^{*''} = A(x + 2)e^x \Rightarrow \L y_2^* = -Ae^x$

        Слагаемые вида $xe^x$ сокращаются (и это неслучайно)

        \textbf{Ответ:} $y = C_1e^{2x} + C_2e^x + \left( \frac{1}{2}x^2 - \frac{1}{2}x + \frac{1}{4} \right)e^{3x} + xe^x$
    \end{enumerate}
\end{Example}

\begin{Exercise}{}
    Сформулировать и обосновать принцип неопределенных коэффициентов для неоднородного уравнения Эйлера 

    \textbf{Подсказка:} Какая там была замена переменных?
\end{Exercise}

\newpage 

\section{Лекция 8. Линейные системы обыкновенных дифференциальных уравнений}

\begin{defin}{Линейные системы обыкновенных дифференциальных уравнений}
    Пусть $t, x_1, \ldots, x_n$ -- вещественные переменные, $a_{11}(t), a_{12}(t), \ldots, a_{nn}(t), f_1(t), \ldots, f_n(t)$ -- непрерывные функции на $I := \q{\alpha, \beta}$, где $\alpha$ и $\beta$ -- конечные или бесконечные величины 

    Линейная неоднородная система обывновенных дифференциальных уравнений это 

    \[ \begin{cases}
        x_1' = a_{11}(t)x_1 + \ldots + a_{1n}(t)x_n + f_1(t) \\
        \vdots \\
        x_n' = a_{n1}(t)x_1 + \ldots + a_{nn}(t)x_n + f_n(t)
    \end{cases} \]

    Система называется однородной, если все $f_i$ равны нулю
\end{defin}

\begin{nota}{Векторная запись систем}
    Положим $x = \begin{pmatrix}
        x_1 \\
        \vdots \\
        x_n
    \end{pmatrix},\ f = \begin{pmatrix}
        f_1 \\
        \vdots \\
        f_n
    \end{pmatrix},\ A = \begin{pmatrix}
        a_{11} & \ldots & a_{1n} \\
        \vdots & \ddots & \vdots \\
        a_{n1} & \ldots & a_{nn}
    \end{pmatrix}$

    Тогда систему можно переписать в виде $x' = A(t)x + f(t)$ (или $x' = A(t)x$ для однородной системы)

    Говорим, что $C^1$ гладкая функция $x(t)$ -- решение уравнения, если $x'(t) = A(t)x(t) + f(t)$
\end{nota}

\begin{defin}{Задача Коши}
    Пусть $x_0 \in \R^n,\ t_0 \in I$. Можем рассмотреть задачу Коши с начальными условиями $(t_0, x_0)$: $\begin{cases}
        x' = A(t)x + f(t) \\
        x(t_0) = x_0
    \end{cases}$
\end{defin}

\begin{theo}{}
    Для любых $x_0 \in \R^n,\ t_0 \in I$ задача Коши имеет единственное решение, которое определено на всем отрезке $I$
\end{theo}

\textit{Доказательство:}

Следует из теорем существования, единственности и продолжимости

\begin{defin}{Пространство решений линейной однородной системы}
    Решения однородной системы образуют линейное пространство 

    $x' = A(t)x,\ y' = A(t)y,\ z = ax + by \Rightarrow z' = A(t)z$
\end{defin}

\begin{Remark}{}
    Существует канонический изоморфизм между решениями и их начальными данными. Соответственно, пространство решений имеет размерность $n$

    Поэтому существует базис из $n$ решений (векторных функций) $\varphi_1(t), \ldots, \varphi_n(t)$
\end{Remark}

\begin{defin}{Фундаментальная матрица системы}
    Общее решение можно записать в виде $x(t) = c_1\varphi_1(t) + \ldots + c_n\varphi_n(t) = \Phi(t)C$, где $\Phi(t) = \begin{pmatrix}
        \varphi_1(t), \ldots, \varphi_n(t)
    \end{pmatrix},\ C = \begin{pmatrix}
        c_1 \\
        \vdots \\
        c_n
    \end{pmatrix}$

    Такая матрица $\Phi$ называется фундаментальной
\end{defin}

\begin{theo}{Свойства фундаментальных матриц}
    Пусть $\Phi(t)$ -- фундаментальная матрица. Тогда $\Phi'(t) = A(t)\Phi(t)$. Это потому что каждый столбец матрицы $\Phi$ удовлетворяет условию $\varphi_k'(t) = A(t)\varphi_k(t)$ и $\varphi_{ki}'(t) = \sum\limits_{j = 1}^n a_{ij}(t)\varphi_{kj}(t)$ 
\end{theo}

\begin{lem}{}
    Если $\Phi(t)$ -- фундаментальная матрица, $B$ -- постоянная невырожденная матрица, то $\Psi(t) = \Phi(t)B$ -- тоже фундаментальная матрица
\end{lem}

\textit{Доказательство:}

Как дифференцировать произведение матриц? 

$\frac{d}{dt}(X(t)Y(t)) = X'(t)Y(t) + X(t)Y'(t)$

$\Psi'(t) = \Phi'(t)B = A(t)\Phi(t)B = A(t)\Psi(t)$

Наоборот, если $\Psi(t), \Psi(t)$ -- фундаментальные матрицы, то существует невырожденная постоянная матрица $B$, такая что $\Psi(t) = \Phi(t)B$

Как считать производные обратных матриц? 

$X^{-1}X(t) = \Id \Rightarrow \frac{d}{dt}X^{-1}(t)X(t) + X^{-1}(t)X'(t) = 0 \Rightarrow \frac{d}{dt}X^{-1}(t) = -X^{-1}(t)X'(t)X^{-1}(t)$

Итак, $\frac{d}{dt}(\Phi^{-1}\Psi) = \frac{d}{dt}\Phi^{-1}\Psi + \Phi^{-1}\Psi' = -\Phi^{-1}\Phi'\Phi^{-1}\Psi + \Phi^{-1}\Psi' = -\Phi^{-1}A\Phi\Phi^{-1}\Psi + \Phi^{-1}A\Psi = 0$

Отсюда матрица $B$ постоянна, а невырождена она, как матрица перехода между двумя базисами

\begin{defin}{Определитель Вронского (вронскиан)}
    Пусть $\varphi_1(t), \ldots, \varphi_n(t)$ -- решения рассматриваемой системы уравнений (необязательно базис), $\Phi(t) = \begin{pmatrix}
        \varphi_1(t), \ldots, \varphi_n(t)
    \end{pmatrix}$

    Тогда $W(t) := \det \Phi(t)$ называется определителем Вронского или вронскианом
\end{defin}

\begin{lem}{Основное свойство вронскиана}
    Для каждого $t_0 \in I$ следующие утверждения равносильны 

    \begin{enumerate}
        \item Решение $\varphi_1(t), \ldots, \varphi_n(t)$ линейно зависимы как функции 
        \item $W(t) = 0$ для каждого $t \in I$
        \item Существует $t_0 \in I$ такое, что $W(t_0) = 0$
    \end{enumerate}
\end{lem}

\textit{Доказательство:}

$1 \Rightarrow 2 \Rightarrow 3$ очевидно. Докажем $3 \Rightarrow 1$

Фиксировав $t_0$, берем постоянные $c_1, \ldots, c_n$ (не все нули) такие, что $\sum\limits_{k = 1}^n c_k \varphi_k(t_0) = 0$

Затем полагаем $\psi(t) := \sum\limits_{k = 1}^n c_k \varphi_k(t)$. Имеем $\psi(t_0) = 0$

С другой стороны, $x_0$ -- это решение задачи Коши. Из единственности решений, $\psi(t) \equiv 0$

\begin{lem}{Дифференцирование определителя}
    $\frac{d}{dt}\begin{vmatrix}
        x_{11} & \ldots & x_{1n} \\
        \vdots & \ddots & \vdots \\
        x_{n1} & \ldots & x_{nn}
    \end{vmatrix} = \begin{vmatrix}
        x_{11}' & \ldots & x_{1n}' \\
        \vdots & \ddots & \vdots \\
        x_{n1} & \ldots & x_{nn}
    \end{vmatrix} + \ldots + \begin{vmatrix}
        x_{11} & \ldots & x_{1n} \\
        \vdots & \ddots & \vdots \\
        x_{n1}' & \ldots & x_{nn}'
    \end{vmatrix}$
\end{lem}

\textit{Доказательство:}

Это следует из формулы $\begin{vmatrix}
    x_{11} & \ldots & x_{1n} \\
    \vdots & \ddots & \vdots \\
    x_{n1} & \ldots & x_{nn}
\end{vmatrix} = \sum\limits_{p \in \prod_n} (-1)^{|p|} \prod\limits_{j = 1}^n x_{j p(j)}$

\begin{theo}{Теорема Остроградского-Лиувилля}
    $W(t) = W(t_0)e^{\int\limits_{t_0}^t \Tr A(s) ds}$
\end{theo}

\textit{Доказательство:}

Пусть $\Phi(t) = \begin{pmatrix}
    \varphi_1(t), \ldots, \varphi_n(t)
\end{pmatrix} = \begin{pmatrix}
    \varphi_{11}(t) & \ldots & \varphi_{1n}(t) \\
    \vdots & \ddots & \vdots \\
    \varphi_{n1}(t) & \ldots & \varphi_{nn}(t)
\end{pmatrix}$. Тогда 

$W'(t) = \begin{vmatrix}
    \varphi_{11}'(t) & \ldots & \varphi_{1n}(t) \\
    \vdots & \ddots & \vdots \\
    \varphi_{n1}(t) & \ldots & \varphi_{nn}(t)
\end{vmatrix} + \ldots + \begin{vmatrix}
    \varphi_{11}(t) & \ldots & \varphi_{1n}(t) \\
    \vdots & \ddots & \vdots \\
    \varphi_{n1}'(t) & \ldots & \varphi_{nn}(t)
\end{vmatrix}$

С другой стороны

$\begin{vmatrix}
    \varphi_{11}'(t) & \ldots & \varphi_{1n}(t) \\
    \vdots & \ddots & \vdots \\
    \varphi_{n1}(t) & \ldots & \varphi_{nn}(t)
\end{vmatrix} = \begin{vmatrix}
    \sum\limits_{j = 1}^n a_{1j}(t)\varphi_{j1}(t) & \ldots & \sum\limits_{j = 1}^n a_{1j}(t)\varphi_{jn}(t) \\
    \vdots & \ddots & \vdots \\
    \varphi_{n1}(t) & \ldots & \varphi_{nn}(t)
\end{vmatrix} = \sum\limits_{j = 1}^n a_{1j} \begin{vmatrix}
        \varphi_{j1}(t) & \ldots & \varphi_{jn}(t) \\
        \vdots & \ddots & \vdots \\
        \varphi_{n1}(t) & \ldots & \varphi_{nn}(t)
\end{vmatrix} = a_{11}(t)W(t)$

С остальными слагаемыми поступаем аналогично

В конечном итоге $W'(t) = \sum\limits_{j = 1}^n a_{jj}(t)W(t) = \Tr A(t) W(t) \Rightarrow W(t) = W(t_0)e^{\int\limits_{t_0}^t \Tr A(s) ds}$

\begin{defin}{Экспонента матрицы}
    Пусть $A$ -- квадратная матрица размера $n \times n$. Тогда экспонентой матрицы $A$ называется сумма ряда $e^A := \Id + \sum\limits_{n = 1}^\infty \frac{A^n}{n!}$, где $A^n$ -- $n$-ая степень матрицы $A$, а $\Id$ -- единичная матрица
\end{defin}

\begin{Remark}{}
    Аналогично можно определить любую аналитическую функцию от матрицы, например, синус. Однако непонятно, почему ряды сходятся
\end{Remark}

\begin{defin}{Матричная норма}
    Пусть $\parl{\cdot}$ -- векторная норма
    
    Определим матричную норму по формуле $\parl{A} = \max\limits_{\parl{x} = 1} \parl{Ax} = \max\limits_{x \neq 0} \frac{\parl{Ax}}{\parl{x}}$
\end{defin}

\begin{theo}{Свойства матричной нормы}
    \begin{enumerate}
        \item $\parl{A} \geq 0;\ \parl{A} = 0 \Leftrightarrow A = 0$
        \item $\parl{\alpha A} = |\alpha| \parl{A}$
        \item $\parl{A + B} \leq \parl{A} + \parl{B}$
        \item $\parl{AB} \leq \parl{A}\parl{B}$
        
        $\parl{AB} = \max\limits_{x \neq 0} \frac{\parl{ABx}}{\parl{x}} = \max\limits_{x \neq 0} \frac{\parl{ABx}}{\parl{Bx}} \cdot \frac{\parl{Bx}}{\parl{x}} \leq \parl{A} \parl{B}$

        \item[4'.] $\parl{A^k} \leq \parl{A}^k;\ \parl{\Id} = 1 \Rightarrow \parl{A^{-1}}\parl{A} \geq 1$
    \end{enumerate}
\end{theo}

\begin{Example}{}
    \begin{enumerate}
        \item $\parl{x} = \max|x_k| \Rightarrow \parl{A} = \max\limits_j \sum\limits_{i = 1}^n |a_{ij}|$ 
        \item $\parl{x} = \sum\limits_{k = 1}^n |x_k| \Rightarrow \parl{A} = \max\limits_i \sum\limits_{j = 1}^n |a_{ij}|$
        \item $\parl{x} = \sqrt{\sum\limits_{k = 1}^n |x_k|^2} \Rightarrow \parl{A} = \max \sqrt{\lambda^*}$, где $\lambda^*$ -- собственные числа матрицы $AA^*$
    \end{enumerate}
\end{Example}

\begin{nota}{}
    Ряд из матриц сходится, если он сходится покоээффициентно
\end{nota}

\begin{propos}{}
    Матричный ряд $\sum\limits_{k = 1}^\infty A_k$ сходится, если сходится ряд из норм $\sum\limits_{k = 1}^\infty \parl{A_k}$
\end{propos}

\begin{lem}{}
    Матричный ряд $e^A$ сходится абсолютно, причем $\parl{e^A} \leq e^{\parl{A}}$
\end{lem}

\textit{Доказательство:}

Доказывается оценкой конечных сумм ряда $e^A = \Id + \sum\limits_{n = 1}^\infty \frac{A^n}{n!}$

\begin{theo}{Свойства экспонент матриц}
    \begin{enumerate}
        \item $e^0 = \Id;\ e^{kA} = (e^A)^k$. В частности $e^{-A} = (e^A)^{-1}$
        \item $AB = BA \Rightarrow e^{A + B} = e^Ae^B = e^Be^A$
        
        \textit{Доказательство:} 

        $e^{A + B} = \Id + \sum\limits_{n = 1}^\infty \frac{(A + B)^n}{n!} = \Id + \sum\limits_{n = 1}^\infty \frac{\sum\limits_{k = 0}^n C_n^k A^kB^{n - k}}{n!} = \Id + \sum\limits_{k, n = 1}^\infty \frac{A^kB^n}{k!n!} = e^Ae^B$
    \end{enumerate}
\end{theo}

\begin{nota}{}
    Пусть $L(\R^n, \R^n)$ -- пространство квадратных матриц размера $n \times n$
\end{nota}

\begin{theo}{}
    Для каждой матрицы $A$ матричная функция $t \mapsto e^{tA} \in L(\R^n, \R^n)$ дифференцируема и удовлетворяет соотношению $\frac{d}{dt}X = AX$
\end{theo}

\textit{Доказательство:}

$(e^{At})' = (\Id + \sum\limits_{k = 1}^\infty \frac{A^kt^k}{k!})' = \sum\limits_{k = 1}^\infty \frac{A^kt^{k - 1}}{(k - 1)!} = Ae^{At}$

\begin{theo}{Следствие}
    Одна из фундаментальных матриц системы имеет вид $X(t) = e^{At}$
\end{theo}

\begin{Remark}{}
    Это не работает, если матрица $A$ зависит от $t$
\end{Remark}

\begin{Exercise}{}
    \begin{enumerate}
        \item Покажите, что матричные синус, косинус и экспонента связаны соотношением $e^{iA} = \cos A + i \sin A$
        \item Покажите, что матрицы $X_1(t) = \sin At$ и $X_2(t) = \cos At$ удовлетворяют соотношению $X'' + A^2X = 0$
    \end{enumerate}
\end{Exercise}

\end{document}