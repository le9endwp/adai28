\documentclass[12pt]{article}
\usepackage{config}
\usepackage{subfiles}
\pgfplotsset{compat=1.18}

\begin{document}

\begin{flushright}
    Конспект Шорохова Сергея

    Если нашли опечатку/ошибку - пишите @le9endwp 
\end{flushright}

\tableofcontents
\newpage

\section{Оргинфа}

Ведет Крыжевич Сергей Геннадьевич

+79219181076 и +48572768176

kryzhevicz@gmail.com и serkryzh@pg.edu.pl

\section{Дифференциальные уравнения первого порядка}

\begin{defin}{Дифференциальные уравнения первого порядка}
    $D \subset \R^2$ -- область, $f : D \to \R$ -- непрерывная функция

    Дифференциальные уравнения первого порядка -- это уравнения вида $y' = f(x, y)$
\end{defin}

\begin{Example}{}
    $y' = xy$
\end{Example}

\begin{defin}{Решение дифференциального уравнения}
    $\q{a, b}$ -- интервал

    Функция $\varphi(x)$ -- решение дифференциального уравнения на $\q{a, b}$, если 

    \begin{enumerate}
        \item $\varphi, \varphi'$ -- непрерывны на $\q{a, b}$
        \item $(x, \varphi(x)) \in D\ \forall x \in \q{a, b}$
        \item $\varphi'(x) = f(x, \varphi(x))$
    \end{enumerate}
\end{defin}

\begin{Example}{}
    $y' = xy$

    Решениями будут:

    \begin{enumerate}
        \item $y = 0$
        \item $y = e^{\frac{x^2}{2}}$
        
        $y' = xe^{\frac{x^2}{2}} = xy$
    \end{enumerate}

    На самом деле решением будет любая функция вида $y = Ce^{\frac{x^2}{2}}$
\end{Example}

\begin{nota}{Начальные данные для дифференциального уравнения}
    $\begin{cases}
        y' = f(x, y) \\
        y(x_0) = y_0
    \end{cases}$
\end{nota}

\begin{defin}{Задача Коши}
    Задача Коши -- дифференциальное уравнение с начальными данными
\end{defin}

\begin{Example}{}
    $\begin{cases}
        y' = xy \\
        y(0) = 5
    \end{cases}$

    $y = Ce^{\frac{x^2}{2}}$

    $5 = Ce^0 = C$

    Получаем ответ $y = 5e^{\frac{x^2}{2}}$
\end{Example}

\begin{defin}{Общее решение дифференциального уравнения}
    Общее решение дифференциального уравнения -- совокупность всех его решений (= решение с параметром)
\end{defin}

\begin{defin}{Интегральная кривая}
    Интегральная кривая -- график решения дифференциального уравнения, т.е. график $\{x, \varphi(x)\}$
\end{defin}

\begin{Remark}{}
    $y' = \sqrt{y};\ y \geq 0$

    Здесь множество не является открытым, но считается, что $y = 0$ является решением (хотя формально им не является)

    Если в каких-то задачах такое будет, в рамках курса не считаем это ошибкой
\end{Remark}

\begin{Remark}{Единственность решений задачи Коши}
    Почти всегда задача Коши имеет единственное решение. Но есть исключения, например

    $\begin{cases}
        y' = 3y^{\frac{2}{3}} \\
        y(0) = 0
    \end{cases}$

    Очевидное решение $y = 0$, но также $y = x^3$. Более того, решением будет любая функция вида $y = (x + C)^3$. График есть на записи

    Более того, можно собрать решение покусочно (ветка параболки вниз + прямая $y = 0$ + ветка параболы вверх)
\end{Remark}

\begin{defin}{Точка единственности/ветвления}
    $\begin{cases}
        y' = f(x, y) \\
        y(x_0) = y_0
    \end{cases}$

    Точка $(x_0, y_0)$ -- точка единственности, если решение задачи Коши единственно. В противном случае это точка ветвления 
\end{defin}

\begin{defin}{Особое решение}
    Решение называется особым, если любая его точка -- точка ветвления
\end{defin}

\begin{theo}{}
    Если в уравнении $y' = f(x, y)$ функция $f$ непрерывна и имеет непрерывную производную по переменной $y$ в области $D$, то для любой точки $(x_0, y_0)$ из $D$ решение задачи Коши с начальными данными $y(x_0) = y_0$ существует и единственно
\end{theo}

\begin{Remark}{}
    По $x$ нужна только непрерывность, производной существовать не обязательно
\end{Remark}

\begin{defin}{Дифференциальные уравнения в симметричной форме}
    $P(x, y)dx + Q(x, y)dy = 0$
\end{defin}

\begin{Example}{}
    $ydx - xdy = 0 \mapsto y' = \frac{y}{x}$ или $x' = \frac{x}{y}$
\end{Example}

\begin{Remark}{}
    Предполагаем, что $P$ и $Q$ -- функции, непрерывные в некоторой области $D$ на плоскости и они не обращаются в ноль одновременно ни в одной точке $D$
\end{Remark}

\begin{defin}{Решение уравнения в симметричной форме}
    \begin{enumerate}
        \item $y' = -\frac{P(x, y)}{Q(x, y)}$, решением будет $y = \varphi(x) : P(x, \varphi(x)) + Q(x, \varphi(x))\varphi'(x) = 0$
        \item $x' = -\frac{Q(x, y)}{P(x, y)}$, решением будет $x = \psi(y) : P(\psi(y), y)\psi'(y) + Q(\psi(y), y) = 0$
        \item $y = \varphi(t), x = \psi(t)$, хотим $P(\psi(t), \varphi(t))\psi'(t) + Q(\psi(t), \varphi(t))\varphi'(t) = 0$
    \end{enumerate}
\end{defin}

\begin{defin}{Системы обыкновенных дифференциальных уравнений}
    $t, x_1, \ldots, x_n \in \R;\ t$ -- время, $x_1 \ldots x_n$ -- фазовые переменные

    $\begin{cases}
        x_1 = f_1(t, x_1, \ldots, x_n) \\
        \ldots \\
        x_n = f_n(t, x_1, \ldots, x_n)
    \end{cases}$ -- скалярная запись системы

    $x = \begin{pmatrix}
        x_1 \\
        \ldots \\
        x_n
    \end{pmatrix};\ f = \begin{pmatrix}
        f_1 \\
        \ldots \\
        f_n
    \end{pmatrix}$ -- векторная запись системы
\end{defin}

\begin{nota}{Как свести уравнение высшего порядка к системам?}
    Пусть есть уравнение $x^{(n)} = g(t, x, x', \ldots, x^{(n-1)})$

    Полагаем $x_1 = x, \ldots, x_n = x^{(n-1)}$

    Получаем $\begin{cases}
        x_1' = x_2 \\
        \ldots \\
        x_{n-1}' = x_n \\
        x_n' = g(t, x_1, \ldots, x_n)
    \end{cases}$
\end{nota}

\begin{Example}{}
    $x'' + \sin x = 0$

    $\begin{cases}
        x_1' = x_2 \\
        x_2' = -\sin x_1
    \end{cases}$
\end{Example}

\begin{Remark}{}
    Предполагается что $f : D \to \R^n$ -- непрерывна и $D \subset \R^{n + 1}$
\end{Remark}

\begin{defin}{Решение системы}
    Функция $\varphi : \q{\alpha, \beta} \to \R^n$ называется решением системы если 

    \begin{enumerate}
        \item $\varphi \in C^1$
        \item $(t, \varphi(t)) \in D\ \forall t \in \q{\alpha, \beta}$
        \item $\varphi(t) = f(t, \varphi(t))\ \forall t \in \q{\alpha, \beta}$
    \end{enumerate}
\end{defin}

\begin{defin}{Задача Коши для систем}
    Пусть $t_0, x_{01}, \ldots, x_{0n} \in \R;\ (t_0, x_{01}, \ldots, x_{0n}) \in D$

    Начальные условия: $\begin{cases}
        x_1(t_0) = x_{01} \\
        \ldots \\
        x_n(t_0) = x_{0n}
    \end{cases}$

    Или в векторной форме: $x(t_0) = x_0$, где $x_0 = \begin{pmatrix}
        x_{01} \\
        \ldots \\
        x_{0n}
    \end{pmatrix}$

    Задача Коши -- уравнение + начальные условия

    $\begin{cases}
        x' = f(t, x) \\
        x(t_0) = x_0
    \end{cases}$
\end{defin}

\begin{defin}{Эквивалентное интегральное уравнение}
    $x(t) = x_0 + \int\limits_{t_0}^t f(x, x(s))ds$

    Функция $\varphi : \q{\alpha, \beta} \to \R^n$ называется решением эквивалентного интегрального уравнения, если

    \begin{enumerate}
        \item $\varphi$ -- непрерывна
        \item $(t, \varphi(t)) \in D\ \forall t \in \q{\alpha, \beta}$
        \item $\varphi(t) = x_0 + \int\limits_{t_0}^t f(s, \varphi(s))ds\ \forall t \in \q{\alpha, \beta}$
    \end{enumerate}
\end{defin}

\begin{lem}{}
    Функция $\varphi(t)$ -- решение задачи Коши тогда и только тогда, когда она является решением эквивалентного интегрального уравнения 
\end{lem}

\textit{Доказательство:}

\begin{itemize}
    \item[$\Rightarrow$] Пусть $\varphi(t)$ -- решение задачи Коши

        \begin{enumerate}
            \item $\varphi$ непрерывна -- очевидно 
            \item $(t, \varphi(t)) \in D\ \forall t \in \q{\alpha, \beta}$ -- то же условие
            \item $\varphi(t) = x_0 + \int\limits_{t_0}^t f(s, \varphi(s))ds\ \forall t \in \q{\alpha, \beta}$ -- получается интегрированием уравнения \\
            $\varphi'(t) = f(t, \varphi(t))$ с учетом начальных условий
        \end{enumerate}
    \item[$\Leftarrow$] Пусть $\varphi(t)$ -- решение интегрального уравнения
    
        \begin{enumerate}
            \item $\varphi$ непрерывна и есть интеграл от непрерывной функции -- значит дифференцируема
            \item $(t, \varphi(t)) \in D\ \forall t \in \q{\alpha, \beta}$ -- то же условие
            \item $\varphi'(t) = f(t, \varphi(t))\ \forall t \in \q{\alpha, \beta}$ -- получается дифференцированием интегрального уравнения
        \end{enumerate}
\end{itemize}

\begin{theo}{Теорема существования решений}
    Пусть правая часть $f(t, x)$ системы $x' = f(t, x)$ непрерывна в области $D \subset \R^{n + 1}$. Пусть $(t_0, x_0) \in D$. Тогда существует решение задачи Коши 

    $\begin{cases}
        x' = f(t, x) \\
        x(t_0) = x_0
    \end{cases}$

    определенное на промежутке $[t_0 - h, t_0 + h]$
\end{theo}

\begin{Remark}{}
    Этот промежуток называется промежутком Пеано
\end{Remark}

\textit{Доказательство:}

Будем вместо решения задачи Коши искать решение эквивалентного интегрального уравнения $x(t) = x_0 + \int\limits_{t_0}^t f(s, x(s))ds$

Поскольку $D$ -- область (открытое множество), выберем константы $a, b > 0$ такие, что $K := \{(t, x) : |t - t_0| \leq a;\ |x - x_0| \leq b\} \subset D$

$K$ -- компакт, значит непрерывная функция огр. Пусть $M = \max\limits_{(t, x) \in K} |f(t, x)|;\ h := \min(a, \frac{b}{M})$

\begin{Remark}{}
    Длина промежутка Пеано непрерывно зависит от начальной точки 
\end{Remark}

\begin{defin}{Векторные нормы}
    Понятие нормы в $\R^n$:

    \begin{enumerate}
        \item $\parl{x} \geq 0;\ \parl{x} = 0 \Leftrightarrow x = 0$
        \item $\parl{ax} = |a|\parl{x}\ \forall a \in \R, x \in \R^n$
        \item $\parl{x + y} \leq \parl{x} + \parl{y}\ \forall x, y \in \R^n$
    \end{enumerate}
\end{defin}

\begin{Example}{}
    \begin{enumerate}
        \item $\parl{x}_1 = |x| = \max(|x_1|, \ldots, |x_n|)$ -- с этой нормой и будем работать
        \item $\parl{x}_2 = \sqrt{x_1^2 + \ldots + x_n^2}$ -- евклидова норма
        \item $\parl{x}_3 = |x_1| + \ldots + |x_n|$
    \end{enumerate}
\end{Example}

\begin{defin}{Равностепенная непрерывность}
    Последовательность функций $\varphi_k : [\alpha, \beta] \to \R^n,\ k \in \N$ -- равностепенно непрерывна, если 
    
    $\forall \varepsilon > 0\ \exists \delta > 0 : \forall t_1, t_2 \in [\alpha, \beta],\ k \in \N$ верно $|t_1 - t_2| < \delta \Rightarrow |\varphi_k(t_1) - \varphi_k(t_2)| < \varepsilon$
\end{defin}

\begin{defin}{Равномерная ограниченность}
    Последовательность функций $\varphi_k : [\alpha, \beta] \to \R^n,\ k \in \N$ -- равномерно ограничена, если

    $\exists C > 0 : \forall t \in [\alpha, \beta],\ k \in \N$ верно $|\varphi_k(t)| \leq C$
\end{defin}

\begin{theo}{Теорема Арцела Асколи}
    Пусть последовательность функций $\varphi_k : [\alpha, \beta] \to \R^n,\ k \in \N$ равностепенно непрерывна и равномерно ограничена. Тогда существует равномерно сходящаяся подпоследовательность $\varphi_{n_k} \toto \varphi_*$ на $[\alpha, \beta]$
\end{theo}

\begin{defin}{Кусочно-гладкая функция}
    Функция $\varphi : [\alpha, \beta] \to \R^n$ называется кусочно-гладкой, если она непрерывна, имеет производную везде, кроме конечного числа точек, а в тех точках имеет односторонние пределы 
\end{defin}

\begin{defin}{$\varepsilon$-решение системы}
    Пусть $\varepsilon > 0$. Кусочно-гладкая функция $\varphi : [\alpha, \beta] \to \R^n$ называется $\varepsilon$-решением системы, если 

    \begin{enumerate}
        \item $(t, \varphi(t)) \in D\ \forall t \in [\alpha, \beta]$
        \item $|\varphi'(t) - f(t, \varphi(t))| \leq \varepsilon$ во всех точках, где производная определена
    \end{enumerate}
\end{defin}

\begin{lem}{}
    Пусть $\varepsilon_m \to 0$ и $\varphi_m(t)$ -- последовательность $\varepsilon_m$-решений системы на отрезке $[\alpha, \beta]$, такая, что $\varphi_m(t_0) = x_0;\ |f(t, \varphi_m(t))| \leq M$ и $\varphi_m \toto \varphi_*$. Тогда $\varphi_*$ -- решение задачи Коши
\end{lem}

\textit{Доказательство:}

Пусть $\Delta_m$ -- последовательность функций, заданных формулой 

$\varphi_m(t) = x_0 + \int\limits_{t_0}^t f(s, \varphi_m(s))ds + \Delta_m(t)$

Интегрируя неравенство $|\varphi'(t) - f(t, \varphi(t))| \leq \varepsilon_m$ от $t_0$ до $t$, с учетом того, что $\varphi_m(t_0) = \varphi(t_0) = x_0$, получаем $|\Delta_m(t)| \leq \varepsilon_m(\beta - \alpha)$

Переходя к пределу в первой формуле, получаем, что $\varphi_*(t)$ -- решение эквивалентного интегрального уравнения, а значит, и задачи Коши 

\begin{Remark}{}
    Далее, мы предложим метод построения таких приближенных решений. Мы будем строить эти решения на промежутке $[t_0, t_0 + h]$, построение на промежутке $[t_0 - h, t_0]$ аналогично
\end{Remark}

\begin{defin}{Ломаные Эйлера}
    Фиксируем $m \in \N$. Разделим отрезок $[t_0, t_0 + h]$ на $m$ равных частей: \\ $t_j = t_0 + \frac{hj}{m};\ j = 0, \ldots, m$

    Положим $\varphi_m(t_0) = x_0$ и последовательно определим $\varphi_m(t) = \varphi_m(t_j) + f(t_j, \varphi_m(t_j))(t - t_j)$ при $j = 0, \ldots, m - 1$ и $t \in [t_j, t_{j + 1}]$

    В частности $\varphi_m(t_{j + 1}) = \varphi_m(t_j) + f(t_j, \varphi_m(t_j))\frac{h}{m}$

    Если положить $A_j = (t_j, \varphi_m(t_j))$, то график $\varphi_m(t)$ -- ломаная, соединяющая точки $A_j$
\end{defin}

\begin{propos}{}
    $K := \{(t, x) := |t - t_0| \leq a;\ |x - x_0| \leq b\} \subset D$

    Для любого $m \in \N, t \in [t_0, t_0 + h]$ верно $(t, \varphi_m(t)) \in K$
\end{propos}

\textit{Доказательство:}

\begin{enumerate}
    \item $|t - t_0| \leq h = \min(a, \frac{b}{M})$
    \item $t^* = \min\limits_{t \in [t_0, t_0 + h]} \{|\varphi_m(t) - x_0| \geq b\}$
    
    С другой стороны, $|\varphi_m(t^*) - x_0| = |\varphi_m(t^*) - \varphi_m(t_0)| \leq \int\limits_t^{t^*} |\varphi_m'(s)|ds \leq M(t^* - t_0) \leq Mh \leq b$
\end{enumerate}

\begin{propos}{}
    $\forall \varepsilon > 0\ \exists m_0$, такое что при $m \geq m_0$ функция $\varphi_m$ является $\varepsilon$-решением системы 
\end{propos}

\textit{Доказательство:}

$|\varphi_m(t_1) - \varphi_m(t_2)| \leq M|t_1 - t_2|$

Если $t \in [t_j, t_{j + 1}]$, то $|t - t_j| \leq \frac{h}{m};\ |f(t_j, \varphi_m(t_j)) - f(t, \varphi_m(t))| \xrightarrow[m \to \infty]{} 0$ равномерно по $t$ 

\begin{propos}{}
    Функции $\varphi_m(t)$ равномерно ограничены 
\end{propos}

\textit{Доказательство:}

$|\varphi_m(t)| \leq M|t - t_0| + |x_0| \leq Mh + |x_0|$

\begin{propos}{}
    Функции $\varphi_m(t)$ равностепенно непрерывны
\end{propos}

\textit{Доказательство:}

$|\varphi_m(t_1) - \varphi_m(t_2)| \leq M|t_1 - t_2|$

\begin{theo}{Теорема Кнезера}
    В условиях теоремы существования, для любого $t_1 \in [t_0 - h, t_0 + h]$ множество значений решений задачи Коши $\{x(t_1) : x(t)\text{ -- решение}\}$ замкнуто и связно
\end{theo}

\begin{Exercise}{}
    Доказать замкнутость (пользуемся утверждениями 2.1-2.4, леммой 2.1 и теоремой Арцела-Асколи)
\end{Exercise}

\begin{lem}{Лемма Гронуолла-Беллмана}
    Пусть $u(t) \geq 0;\ f(t) \geq 0; u(t), f(t) \in C[t_0, \infty)$, при этом \\
    для $t \geq t_0$ выполняется неравенство $u(t) \leq c + \int\limits_{t_0}^t f(t_1)u(t_1)dt_1$, где $c > 0$ -- константа

    Тогда при $t \geq t_0$ имеем оценку $u(t) \leq c \cdot \exp(\int\limits_{t_0}^t f(t_1)dt_1)$
\end{lem}

\textit{Доказательство:}

Из неравенства получаем $\frac{u(t)}{c + \int\limits_{t_0}^t f(t_1)u(t_1)dt_1} \leq 1$ и $\frac{f(t)u(t)}{c + \int\limits_{t_0}^t f(t_1)u(t_1)dt_1} \leq f(t)$

Т.к. $\frac{d}{dt}\left[ c + \int\limits_{t_0}^t f(t_1) u(t_1)dt_1 \right] = f(t)u(t)$, то проинтегрировав от $t_0$ до $t$, получим 

$\ln\left[ c + \int\limits_{t_0}^t f(t_1)u(t_1)dt_1 \right] - \ln c \leq \int\limits_{t_0}^t f(t_1)dt_1$, отсюда и из неравенства 

$u(t) \leq c + \int\limits_{t_0}^t f(t_1)u(t_1)dt_1 \leq c \cdot \exp(\int\limits_{t_0}^t f(t_1)dt_1)$, чтд

\begin{theo}{Следствие леммы Гронуолла-Беллмана}
    \begin{enumerate}
        \item $u(t) \leq \int\limits_{t_0}^t f(t_1)u(t_1)dt_1 \Rightarrow u(t) \equiv 0$
        \item $t \leq t_0$
        
        $u(t) \leq c + \left| \int\limits_{t_0}^t f(t_1)u(t_1)dt_1 \right| \Rightarrow u(t) \leq c \cdot \exp\left| \int\limits_t^{t_0} f(t_1)dt_1 \right|$
    \end{enumerate}
\end{theo}

\textit{Доказательство:}

В пункте 2 замена $s = -t$ 

\begin{lem}{Усиленная лемма Гронуолла-Беллмана}
    Пусть функция $u(x)$ неотрицательна и непрерывна в промежутке $[x_0, x_0 + h]$ и удовлетворяет там неравенству $0 \leq u(x) \leq A + B \int\limits_{x_0}^x u(t)dt + \varepsilon(x - x_0)$ при $A, B, \varepsilon \geq 0$

    Тогда при $x \in [x_0, x_0 + h]$ справедливо неравенство $u(x) \leq Ae^{B(x - x_0)} + \frac{\varepsilon}{B}(e^{B(x - x_0)} - 1)$
\end{lem}

\begin{Exercise}{}
    Доказать усиленную лемму Гронуолла-Беллмана
\end{Exercise}

\begin{defin}{Условие Липшица}
    Непрерывная функция (вектор-функция) $f : A \mapsto \R^n$ удовлетворяет условию Липшица, $f \in Lip(A)$, если существует такая константа $L > 0$, что $|f(x) - f(y) \leq L|x - y|$ для любых $x, y \in A$
\end{defin}

\begin{defin}{Локальное условие Липшица}
    Непрерывная функция (вектор-функция) $f : A \mapsto \R^n$ удовлетворяет локальному условию Липшица, если для любого $x_0 \in A$ существует окрестность $U$ точки $x_0$, в которой функция $f$ удовлетворяет условию Липшица
\end{defin}

\begin{lem}{}
    Функция $f : U \to \R^n$ -- непрерывно дифференцируема, где $U$ -- область в $\R^m$, значит $f$ удовлетворяет в этой области локальному условию Липшица
\end{lem}

\textit{Доказательство:}

Возьмем точку $x_0 \in U$ и замкнутый шарик $B$ с центром в $x_0$ такой, что $B \subset U$. Пусть $M = \max\limits_{x \in B}|Df(x)|$. Тогда по теореме о среднем $|f(x) - f(y)| \leq M|x - y|$ для любых $x, y \in B$

\begin{defin}{Условие Липшица по переменной $x$}
    Пусть $U \subset \R^{n + 1}_{t, x}$ -- область. Непрерывная вектор-функция $f : U \to \R^n$ удовлетворяет условию Липшица по переменной $x$, $f \in Lip_x(A)$ если существует такая константа $L > 0$, что $|f(t, x_1) - f(t, x_2)| \leq L|x_1 - x_2|$ для любых $(t, x_1), (t, x_2) \in U$
\end{defin}

\begin{defin}{Локальное условие Липшица по переменной $x$}
    Пусть $U \subset \R^{n + 1}_{t, x}$ -- область. Непрерывная вектор-функция $f : U \to \R^n$ удовлетворяет локальному условию Липшица по переменной $x$, $f \in Lip_{loc, x}(A)$, если для любой точки $(t_0, x_0) \in U$ существует окрестность $V$ этой точки и такая константа $L > 0$, что $|f(t, x_1) - f(t, x_2)| \leq L|x_1 - x_2|$ для любых $(t, x_1), (t, x_2) \in V$
\end{defin}

\begin{theo}{Теорема об условии Липшица в компакте}
    Пусть вектор-функция $f$ удовлетворяет локальному условию Липшица по $x$ в области $U$. Тогда для любого компакта $K \subset U$ эта функция липшицева по $x$ на этом компакте
\end{theo}

\textit{Доказательство:}

Пусть это утверждение неверно. Тогда существуют последовательности $(t_k, x_k) \in K$ и \\ 
$(t_k, y_k) \in K, x_k \neq y_k$, такие что $|f(t_k, x_k) - f(t_k, y_k)| \geq k|x_k - y_k|$

НУО можем считать, что $t_k \to t^*,\ x_k \to x^*,\ y_k \to y^*$. При этом $(t^*, x^*), (t^*, y^*) \in K$

Возможны два случая:

\begin{enumerate}
    \item $x^* \neq y^*$

    Тогда $\frac{|f(t_k, x_k) - f(t_k, y_k)|}{|x_k - y_k|} \to \infty,\ |x_k - y_k| \not\to 0 \Rightarrow |f(t_k, x_k) - f(t_k, y_k)|$ не ограничено. Противоречие (т.к. $f$ непрерывна на компакте)

    \item $x^* = y^*$
    
    В этом случае существует окрестность $U$ точки $(t^*, x^*)$ такая, что существует константа $L > 0$, что $|f(t, x) - f(t, y)| \leq L|x - y|$ для любых $(t, x), (t, y) \in U$

    Значит, такое неравенство выполнено для всех $(t_k, x_k), (t_k, y_k)$ начиная с некоторого номера. Противоречие
\end{enumerate}

\begin{theo}{Теорема единственности}
    Пусть $x' = f(t, x)$ -- система оду. $f : D \to \R^n;\ D \subset \R^{n + 1}$ -- область. Пусть $f$ непрерывна и локально липшицева по $x$ в области $D$

    Тогда для любой пары $(t_0, x_0) \in D$ задача Коши $\begin{cases}
        x' = f(t, x) \\
        x(t_0) = x_0
    \end{cases}$ имеет единственное решение
\end{theo}

Пусть утверждение теоремы неверно. Есть такая точка $(t_0, x_0) \in D$, что задача Коши имеет два различных решения $\varphi(t)$ и $\psi(t)$ на промежутке $[t_0 - h, t_0 + h]$

$K = \{(t, \varphi(t)) : t \in [t_0 - h, t_0 + h]\} \cup \{(t, \psi(t)) : t \in [t_0 - h, t_0 + h]\}$

Множество $K$ -- компакт. На нем выполнено глобальное условие Липшица по $x$, в частности 

$|f(t, \varphi(t)) - f(t, \psi(t))| \leq L|\varphi(t) - \psi(t)|$. Положим $u(t) = |\varphi(t) - \psi(t)|$

$\varphi(t) = x_0 + \int\limits_{t_0}^t f(s, \varphi(s))ds;\ \psi(t) = x_0 + \int\limits_{t_0}^t f(s, \psi(s))ds$

$\varphi(t) - \psi(t) = \int\limits_{t_0}^t \left[ f(s, \varphi(s)) - f(s, \psi(s)) \right] ds$

$|\varphi(t) - \psi(t)| = \left| \int\limits_{t_0}^t \left[ f(s, \varphi(s)) - f(s, \psi(s)) \right] ds \right| \leq \int\limits_{t_0}^t |f(s, \varphi(s)) - f(s, \psi(s))| ds \leq L\int\limits_{t_0}^t |\varphi(s) - \psi(s)| ds$

$u(t) \leq L\int\limits_{t_0}^t u(s) ds$ по следствию из леммы Гронуолла-Беллмана $u(t) \equiv 0$ и $\varphi(t) \equiv \psi(t)$

\begin{theo}{Следствие}
    Пусть $x' = f(t, x)$ -- система оду. $f : D \to \R^n;\ D \subset \R^{n + 1}$ -- область. Пусть $f$ непрерывна и непрерывно дифференцируема по $x$ в области $D$. Тогда для любой пары $(t_0, x_0) \in D$ задача Коши $\begin{cases}
        x' = f(t, x) \\
        x(t_0) = x_0
    \end{cases}$ имеет единственное решение
\end{theo}

\begin{Remark}{}
    Условие теоремы единственности достаточное, но не необходимое 

    $y' = y\ln|y|;\ y \neq 0;\ y' = 0$ при $y = 0$

    $y = 0$ или $\ln|\ln|y|| = x + c \Rightarrow y = e^{ce^x}$

    Единственность решений есть, а условия Липшица (даже локального) нет
\end{Remark}

\begin{defin}{Продолжение решения}
    Пусть есть решения $\varphi : \q{a, b} \to \R^n,\ \psi : \q{a_1, b_1} \to \R^n$. Говорим, что решение $\psi$ есть продолжение решения $\varphi$ (продолжает решение $\varphi$), если 

    \begin{enumerate}
        \item $\q{a, b} \not\subseteq \q{a_1, b_1}$
        \item $\psi\mid_{\q{a, b}} = \varphi$
    \end{enumerate}
\end{defin}

\begin{defin}{Продолжимость влево}
    Решение $\varphi(t)$ называется продолжимым влево за $a$, если существует решение $\psi(t)$, продолжающее решение, и при этом $a_1 < a$
\end{defin}

\begin{defin}{Продолжимость вправо}
    Решение $\varphi(t)$ называется продолжимым вправо за $b$, если существует решение $\psi(t)$, продолжающее решение, и при этом $b_1 > b$
\end{defin}

\begin{defin}{Максимально продолженное решение}
    Решение $\varphi(t)$ называется непродолжимым или максимально продолженным, если решения $\psi(t)$, продолжающего $\varphi(t)$, не существует
\end{defin}

\begin{theo}{Теорема о продолжимости решений вправо за $b$}
    Пусть решение $\varphi(t)$ уравнения $x' = f(t, x)$ задано на промежутке $\langle a, b)$, причем существует предел $\lim\limits_{t \to b_-} \varphi(t) = x_0$ и $(b, x_0) \in D$. Тогда решение $\varphi(t)$ продолжимо вправо за $b$

    Теорема о продолжимости решения влево за $a$ выглядит аналогично
\end{theo}


\end{document}

