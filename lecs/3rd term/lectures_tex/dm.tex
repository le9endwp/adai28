\documentclass[12pt]{article}
\usepackage{config}
\usepackage{subfiles}
\pgfplotsset{compat=1.18}

\begin{document}

\begin{flushright}
    Конспект Шорохова Сергея

    Если нашли опечатку/ошибку - пишите @le9endwp 
\end{flushright}

\tableofcontents
\newpage

\section{\S 1. Производящие функции}

\begin{defin}{Производящая функция}
    Пусть $(a_n)_{n = 0}^\infty$ -- последовательность. Ее производящая функция -- это формальный степенной ряд $A(t) = \sum\limits_{n = 0}^\infty a_n t^n = a_0 + a_1t + a_2t^2 + \ldots$
\end{defin}

\begin{nota}{Элементарные операции}
    \begin{enumerate}
        \item $A(t) \pm B(t) = \sum\limits_{n = 0}^\infty (a_n \pm b_n) t^n$
        \item $c \in \C \Rightarrow c \cdot A(t) = \sum\limits_{n = 0}^\infty (ca_n)t^n$
        \item $A(t)B(t) = a_0b_0 + (a_0b_1 + a_1b_0)t + \ldots + (a_0b_n + a_1b_{n - 1} + a_2b_{n - 2} + \ldots + a_nb_0)t^n + \ldots$
    \end{enumerate}
\end{nota}

\begin{defin}{Свертка}
    Последовательность $(c_n)_{n = 0}^\infty$, где $c_n = a_0b_n + a_1b_{n - 1} + \ldots + a_{n - 1}b_1 + a_nb_0$ называется сверткой последовательностей $(a_n)_{n = 0}^\infty$ и $(b_n)_{n = 0}^\infty$
\end{defin}

\begin{Remark}{}
    Множество производящих функций образует коммутативное кольцо с единицей; векторное пространство над полем $\C$

    Вообще это называется коммутативная алгебра с единицей
\end{Remark}

\begin{defin}{Композиция производящих функций}
    Пусть $b_0 = 0$

    $A(B(t)) = a_0 + a_1B(t) + a_2B(t)^2 + \ldots = a_0 + a_1(b_1t + b_2t^2 + b_3t^3 + \ldots) + \\
    + a_2(b_1^2t^2 + 2b_1b_2t^3 + \ldots) + a_3(b_1^3t^3 + \ldots) = a_0 + a_1b_1t + (a_1b_2 + a_2b_1)t^2 + (a_1b_3 + 2a_2b_1b_2 + a_3b_1^3)t^3$

    \begin{Example}{}
        $A(-t) = a_0 - a_1t + a_2t^2 - a_3t^3 + \ldots$
    \end{Example}
\end{defin}

\newpage

\begin{theo}{}
    Пусть $a_0 \neq 0$. Тогда $\exists! B(t)$, т.ч. $A(t) B(t) = 1$
\end{theo}

\textit{Доказательство:}

Ищем $B(t) = b_0 + b_1t + b_2t^2 + \ldots$

$A(t)B(t) = a_0b_0 + (a_0b_1 + a_1b_0)t + (a_2b_0 + a_1b_1 + a_0b_2)t^2 + \ldots = 1$

$a_0b_0 = 1 \Rightarrow$ находим $b_0$

$\underbrace{a_1b_0}_\text{знаем} + a_0b_1 = 0 \Rightarrow$ находим $b_1$

$\underbrace{a_2b_0 + a_1b_1}_\text{знаем} + a_0b_1 = 0 \Rightarrow$ находим $b_2$

И так далее \dots

\begin{theo}{}
    $b_0 = 0, b_1 \neq 0$. Тогда $\exists! A(t)$ и $C(t)$, т.ч. $a_0 = c_0 = 0$ и $A(B(t)) = B(C(t)) = t$
\end{theo}

\begin{Exercise}{}
    Доказать теорему 1.2.
\end{Exercise}

\begin{defin}{Производная}
    $A'(t) = a_1 + 2a_2t + 3a_3t^2 + \ldots = \sum\limits_{n = 1}^\infty na_nt^{n - 1}$

    $t \cdot A'(t) = \sum\limits_{n = 0}^\infty na_nt^n$
\end{defin}

\begin{defin}{Первообразная}
    $\int A(t)dt = a_0t + \frac{a_1}{2}t^2 + \frac{a_2}{3}t^3 + \ldots$
\end{defin}

\begin{Remark}{}
    $(\int A(t)dt)' = A(t);\ \int A'(t)dt = A(t) - a_0$
\end{Remark}

\begin{Example}{}
    \begin{enumerate}
        \item $a_n \equiv 1 \Rightarrow A(t) = \sum\limits_{n = 0}^\infty t^n = \frac{1}{1 - t}$
        
        Пусть $(b_n)_{n = 0}^\infty$ -- произвольная последовательность

        $c_n = \underbrace{b_0 + b_1 + \ldots + b_n}_\text{свертка послед. выше}$; $C(t) = \frac{B(t)}{1 - t}$

        \item $e^t = \sum\limits_{k = 0}^\infty \frac{t^k}{k!}$
        \item $\sum\limits_{n = 0}^\infty t^n = \frac{1}{1 - t};\ \sum\limits_{n = 0}^\infty nt^n = t(\frac{1}{1 - t})' = \frac{t}{(1 - t)^2}$
    \end{enumerate}
\end{Example}

\begin{Example}{Числа Фиббоначи}
    $F_0 = 0,\ F_1 = 1,\ F_{n + 2} = F_{n + 1} + F_n \Rightarrow F_{n + 2}t^{n + 2} = F_{n + 1}t^{n + 2} + F_n t^{n + 2}$

    $F(t) = \sum\limits_{n = 0}^\infty F_n t^n;\ \underbrace{\sum\limits_{n = 0}^\infty F_{n + 2}t^{n + 2}}_{F(t) - t} = \underbrace{\sum\limits_{n = 0}^\infty F_{n + 1}t^{n + 2}}_{t (\sum\limits_{n = 0}^\infty F_{n + 1}t^{n + 1}) = tF(t)} + \underbrace{\sum\limits_{n = 0}^\infty F_nt^{n + 2}}_{t^2F(t)}$

    $F(t) - t = tF(t) + t^2F(t)$

    $F(t) = \frac{t}{1 - t - t^2}$ -- производящая функция для чисел Фиббоначи

    Корни знаменателя $(t_2 + t - 1 = 0 \Leftrightarrow t = \frac{-1 \pm \sqrt{5}}{2});\ \varphi = \frac{1 + \sqrt{5}}{2};\ \psi = \frac{1 - \sqrt{5}}{2}$

    $1 - t - t^2 = (1 - \varphi t)(1 - \psi t)$

    Ищем разложение на простейшие $\frac{t}{1 - t - t^2} = \frac{A}{1 - \varphi t} + \frac{B}{1 - \psi t} \Leftrightarrow \\ 
    \Leftrightarrow t = A(1 - \psi t) + B(1 - \varphi t) \Leftrightarrow \begin{cases}
        A + B = 0 \\
        A \psi + B \varphi = -1
    \end{cases} \Leftrightarrow  \\ \Leftrightarrow \begin{cases}
        B = -A \\
        A \psi - A \varphi = - 1 \Rightarrow  \begin{cases}
            A = \frac{1}{\varphi - \psi} = \frac{1}{\sqrt{5}} \\
            B = - \frac{1}{\sqrt{5}}
        \end{cases}
    \end{cases}$

    Итого: $F(t) = \frac{1}{\sqrt{5}}(\frac{1}{1 - \varphi t} - \frac{1}{1 - \psi t}) = \frac{1}{\sqrt{5}}(\sum\limits_{n = 0}^\infty \varphi^nt^n - \sum\limits_{n = 0}^\infty \psi^nt^n)$

    $F_n = \frac{1}{\sqrt{5}}((\frac{1 + \sqrt{5}}{2})^n - (\frac{1 - \sqrt{5}}{2})^n) \approx \frac{1}{\sqrt{5}}(\frac{1 + \sqrt{5}}{2})^n$
\end{Example}

\begin{nota}{Как решать линейные рекуррентные соотношения?}
    $a_{n + k} = c_1a_{n + k - 1} + c_2a_{n + k - 2} + \ldots + c_ka_n$; знаем $a_0, a_1 \ldots a_{k - 1}$

    $A(t) = \sum\limits_{n = 0}^\infty a_nt^n$

    $a_{n + k}t^{n + k} = c_1ta_{n + k - 1}t^{n + k - 1} + c_2t^2a_{n + k - 2}t^{n + k - 2} + \ldots + c_kt^ka_nt^n$

    Суммируем по $n = 0 : \underbrace{\sum\limits_{n = 0}^\infty a_{n + k}t^{n + k}}_{A(t) - a_0 - a_1t - \ldots - a_{k - 1}t^{k - 1}} = c_1t\underbrace{\sum\limits_{n = 0}^\infty a_{n + k - 1}t^{n + k - 1}}_{A(t) - a_0 - a_1t - \ldots - a_{k - 2}t^{k - 2}} + \ldots + c_kt^k\underbrace{\sum\limits_{n = 0}^\infty a_nt^n}_{A(t)}$

    Получаем уравнение: $(1 - c_1t - c_2t^2 - \ldots - c_kt^k)A(t) = \underbrace{P(t)}_\text{многочлен, знаем}$

    $A(t) = \frac{P(t)}{Q(t)}$ -- рациональная функция 

    $Q(t) = (1 - \alpha_1t)^{r_1} (1 - \alpha_2t)^{r_2} \ldots (1 - \alpha_et)^{r_e}$

    Раскладываем на простейшие вида $\frac{1}{(1 - \alpha_s t)^{m}}$

    $\frac{1}{1 - \alpha_s t} = \sum\limits_{n = 0}^\infty \alpha_s^nt^n$

    $\frac{1}{(1 - \alpha_st)^2} = \sum\limits_{n = 0}^\infty (n + 1)\alpha_s^nt^n$
\end{nota}

\begin{Remark}{Вопрос}
    Когда производящая функция -- рациональная?
\end{Remark}

\begin{defin}{Квазимногочлен}
    Последовательность $(a_n)_{n = 0}^\infty$ -- квазимногочлен, если $a_n = c_1(n)q_1^n + c_2(n)q_2 + \ldots + c_k(n)q_k^n$, где $q_1 \ldots q_k \in \C;\ c_1(n) \ldots c_k(n)$ -- многочлены с комплексными коэффициентами 
\end{defin}

\begin{theo}{}
    $A(t) = \sum\limits_{n = 0}^\infty a_nt^n;\ A(t)$ -- рациональна $\Leftrightarrow (a_n)_{n = 0}^\infty$ -- квазимногочлен, начиная с некоторого места
\end{theo}

\textit{Доказательство:}

\begin{itemize}
    \item["$\Rightarrow$"\ ] $A(t)$ -- рациональная $\Rightarrow$ раскладываем на простейшие вида $(1 - qt)^{-m} + \underbrace{\text{ некоторый многочлен}}_\text{влияет на первые неск. эл. посл-ти}$
    
    $(1 - qt)^{-m} = \sum\limits_{n = 0}^\infty {m + n - 1 \choose n} q^nt^n = \sum\limits_{n = 0}^\infty \underbrace{\frac{(n + 1)(n + 2)\ldots(n + m - 1)}{(m - 1)!}}_\text{многочлен от $n$}q^nt^n$

    \item["$\Leftarrow$"\ ] Надо доказать, что $(c(n)q^n)_{n = 0}^\infty$ имеет рациональную производящую функцию
    
    $c(n) = \sum\limits_{m \geq 0} \alpha_m n(n + 1)\ldots(n + m) = \alpha_0 + \alpha_1(n + 1) + \alpha_2(n + 1)(n + 2) + \ldots$

    $\sum\limits_{n = 0}^\infty c(n)q^nt^n = \sum\limits_{n = 0}^\infty \sum\limits_{m \geq 0} \alpha_m(n + 1)(n + 2) \ldots (n + m) \cdot (qt)^n \stackrel{x = qt}{=} \sum\limits_{m \geq 0} \alpha_m \sum\limits_{n = 0}^\infty \underbrace{(n + 1)(n + 2) \ldots (n + m) x^n}_{(x^{n + m})^{(m)}} = \\
     = \sum\limits_{m \geq 0} \alpha_m \cdot (\sum\limits_{n = 0}^\infty x^{n + m})^{(m)} = \sum\limits_{m \geq 0} \alpha_m (\sum\limits_{n = 0}^\infty x^n)^{(m)} = \sum\limits_{m \geq 0} \alpha_m (\frac{1}{1 - x})^{(m)}$

     Получаем рациональную функцию 
\end{itemize}

\begin{defin}{Произведение Адамара}
    $A(t) = \sum\limits_{n = 0}^\infty a_nt^n;\ B(t) = \sum\limits_{n = 0}^\infty b_nt^n$

    Произведение Адамара $A(t) \odot B(t) = \sum\limits_{n = 0}^\infty (a_nb_n)t^n$
\end{defin}

\begin{theo}{Следствие}
    Произведение Адамара рациональных функций -- рациональная функция (очевидно из теоремы)
\end{theo}

\begin{Example}{}
    $F_1 + \ldots F_n = S_n =\ ?$

    $\F(t) = \sum\limits_{n = 0}^\infty F_nt^n = \underbrace{\frac{t}{1 - t - t^2}}_{\frac{1}{\sqrt{5}}(\frac{1}{1 - \varphi 1} - \frac{1}{1 - \psi t})} = \sum\limits_{n = 0}^\infty S_nt^n = \frac{\F(t)}{1 - t}$

    $S(t) = \frac{1}{\sqrt{5}}(\frac{1}{1 - \varphi t} - \frac{1}{1 - \psi t}) \frac{1}{1 - t}$

    Разложим $\frac{1}{1 - \varphi t} \cdot \frac{1}{1 - t} = \frac{A}{1 - \varphi t} + \frac{B}{1 - t} \Leftrightarrow 1 = A(1 - t) + B(1 - \varphi t) \Leftrightarrow \begin{cases}
        B = \frac{1}{1 - \varphi} = - \varphi \\
        A = 1 + \varphi
    \end{cases}$

    Аналогично $\frac{1}{1 - \psi t} \cdot \frac{1}{1 - t} = \frac{1 + \psi}{1 - \psi t} - \frac{\psi}{1 - t}$

    Итого, $S(t) = \frac{1}{\sqrt{5}}(\underbrace{\frac{1 + \varphi}{1 - \varphi t} - \frac{1 + \psi}{1 - \psi t}}_{\frac{1 + \varphi - \psi t - \varphi \psi t - 1 - \psi + \varphi t + \varphi \psi t}{1 - t - t^2}} - \underbrace{\frac{\varphi - \psi}{1 - t}}_{\frac{\sqrt{5}}{1 - t}})$

    $S(t) = \frac{1 + t}{1 - t - t^2} - \frac{1}{1 - t}$

    $\frac{t}{1 - t - t^2} = \sum\limits_{n = 0}^\infty F_nt^n = \F(t)$

    $\frac{1}{1 - t - t^2} = \frac{\F(t)}{t} = \sum\limits_{n = 1}^\infty F_nt^{n - 1} = \sum\limits_{ n = 0}^\infty F_{n + 1}t^n$

    Ответ: $F_{n + 2} - 1$
\end{Example}

\begin{Example}{Еще один пример}
    \textbf{Осторожно! На записи рисуночки}

    Взаимно рекуррентные последовательности

    \textbf{Задача:} сколько способов разбить прямоугольник $3 \times n$ на доминошки $1 \times 2$?

    $v_n$ -- кол-во способов разбить прямоугольник $3 \times n$ без левой нижней клетки

    $u_n$ -- кол-во способов разбить прямоугольник $3 \times n$

    Методом нехитрого посмотреть запись и увидеть красивые рисунки становится очевидно, что 

    $\begin{cases}
        u_n = 2v_{n - 1} + u_{n - 2} \\
        v_n = u_{n - 1} + v_{n - 2}
    \end{cases}$ при $u_1 = 0, u_2 = F_4 = 3;\ v_1 = 1, v_2 = 0$. Пусть $u_0 = 1;\ v_0 = 0$

    $U(t) = \sum\limits_{n = 0}^\infty u_nt^n;\ V(t) = \sum\limits_{n = 0}^\infty v_nt^n$

    $\begin{cases}
        u_{n + 2}t^{n + 2} = 2v_{n + 1}t^{n + 2} + u_nt^{n + 2} \\
        v_{n + 2}t^{n + 2} = u_{n + 1}t^{n + 2} + v_nt^{n + 2}
    \end{cases} \Rightarrow \begin{cases}
        U(t) - 1 = 2tV(t) + t^2U(t) \\
        V(t) - t = t^2V(t) + t(U(t) - 1)
    \end{cases}$

    $V(t) = \frac{t}{1 - t^2}U(t)$. Подставляем во 2 уравнение

    $U(t) - 1 = \frac{2t^2}{1 - t^2}U(t) + t^2U(t)$

    $U(t) = \frac{1 - t^2}{1 - 4t^2 + t^4}$

    Пусть $t^2 = s$, тогда $W(s) = \frac{1 - s}{1 - 4s + s^2} = \underbrace{\frac{A}{1 - \varphi s}}_{A \cdot \sum\limits_{n = 0}^\infty \varphi^ns^n} + \underbrace{\frac{B}{1 - \psi s}}_{B \cdot \sum\limits_{n = 0}^\infty \psi^ns^n} \Rightarrow \begin{cases}
        A = \frac{1 + \sqrt{3}}{2\sqrt{3}} \\
        B = \frac{\sqrt{3} - 1}{2\sqrt{3}}
    \end{cases}$

    $u_{2n} = A \varphi^n + B \psi^n = \frac{1 + \sqrt{3}}{2\sqrt{3}}(2 + \sqrt{3})^n + \frac{\sqrt{3} - 1}{2\sqrt{3}}(2 - \sqrt{3})^n \approx \frac{1 + \sqrt{3}}{2\sqrt{3}}(2 + \sqrt{3})^n$
\end{Example}

\newpage

\section{\S 2. Биномиальные коэффициенты}

\begin{Reminder}{}
    $C_n^k = \frac{n!}{k!(n - k)!}$

    $C_n^k = {n \choose k} = \frac{n(n - 1)(n - 2) \ldots (n - k + 1)}{k!}$ -- определено при всех $n \in \C$

    $(1 + t)^\alpha = \sum\limits_{k = 0}^\infty {\alpha \choose k} t^k$
\end{Reminder}

\begin{Example}{}
    \begin{enumerate}
        \item $\underbrace{\alpha(1 + t)^{\alpha - 1}}_{\alpha\sum\limits_{k = 0}^\infty {\alpha - 1 \choose k} t^k} = ((1 + t)^\alpha)' = \sum\limits_{k = 1}^\infty {\alpha \choose k}kt^{k - 1}$
        
        Приравниваем коэффициенты при $t^{k - 1}$: $\alpha{\alpha - 1 \choose k - 1} = k {\alpha \choose k}$

        \item $((1 + t)^\alpha)^{(m)} = (\sum\limits_{k = 0}^\infty {\alpha \choose k}t^k)^{(m)} = \sum\limits_{k = m}^\infty {\alpha \choose k} \underbrace{k(k - 1) \ldots (k - m + 1)}_{m!{k \choose m}}t^{k - m}$
        
        $((1 + t)^\alpha)^{(m)} = \underbrace{\alpha(\alpha - 1)\ldots(\alpha - m + 1)}_{m! {\alpha \choose m}}(1 + t)^{\alpha - m} = m! {\alpha \choose m} \sum\limits_{k = 0}^\infty {\alpha - m \choose k} t^k$

        Приравниваем коэффициенты при $t^{k - m}$: ${\alpha \choose m}{\alpha - m \choose k - m} = {\alpha \choose k}{k \choose m}$

        \item $(1 + t)^{2n} = (1 + t)^n(1 + t)^n = (\sum\limits_{k = 0}^n {n \choose k}t^k)^2$
        
        $(1 + t)^{2n} = \sum\limits_{k = 0}^{2n} {2n \choose k}t^k$

        Коэффициент при $t^n$: ${2n \choose n} = {n \choose 0}{n \choose n} + {n \choose 1}{n \choose n - 1} + \ldots + {n \choose n}{n \choose 0}$

        ${2n \choose n} = {n \choose 0}^2 + {n \choose 1}^2 + \ldots + {n \choose n}^2$

        \item $(1 + t)^p(1 + t)^{n - p} = ( 1 + t)^n$
        
        ${n \choose m} = {p \choose 0} {n - p \choose m} + {p \choose 1}{n - p \choose m - 1} + \ldots + {p \choose m}{n - p \choose 0}$ -- свертка Вандермонда

        \item $(1 - t)^{-n - 1} = \sum\limits_{k = 0}^\infty {n + k \choose k} t^k = \sum\limits_{k = 0}^\infty {n + k \choose n} t^k$ -- производящая $({n + k \choose n})_{k = 0}^\infty$
        
        $\frac{1}{(1 - t)^{n + 2}} = \frac{1}{1 - t} \cdot \frac{1}{(1 - t)^{n + 1}}$ -- производящая функция $({n + 0 \choose n} + \ldots + {n + k \choose n})_{k = 0}^\infty$

        Коэффициент при $t^m$: ${n + m + 1 \choose n + 1} = {n + 0 \choose n} + {n + 1 \choose n} + \ldots + {n + m \choose n}$

        $m + 1 = k \Rightarrow {n + k \choose n + 1} = {n \choose n} + {n + 1 \choose n} + \ldots + {n + k - 1 \choose n}$

        $n = 1 \Rightarrow \frac{k(k + 1)}{2} = 1 + 2 + \ldots + k$

        $n = 2 \Rightarrow 1 \cdot 2 + 2 \cdot 3 + \ldots + k(k + 1) = 2{ k + 2 \choose 3} = \frac{k(k + 1)(k + 2)}{3}$
    \end{enumerate}
\end{Example}

\newpage 

\section{\S 3. Числа Каталана}

\begin{defin}{Правильная скобочная последовательность}
    Последовательность из $2n$ открывающих и закрывающих скобок -- правильная, если в любом префиксе открывающих $\geq$ закрывающих и всего $n$ открывающих и $n$ закрывающих
\end{defin}

\begin{defin}{Числа Каталана}
    Числа Каталана ($C_n$) -- количество правильных скобочных последовательностей длины $2n$
\end{defin}

\begin{nota}{}
    $C_0 = 1$

    $C_1 = 1$: () 

    $C_2 = 2$: ()(), (())

    $C_3 = 5$: ((())), (()()), (())(), ()(()), ()()()
\end{nota}

\begin{propos}{}
    $\forall n \geq 1\ C_n = C_0C_{n - 1} + C_1C_{n -2} + \ldots + C_{n - 2}C_1 + C_{n - 1}C_0$
\end{propos}

\textit{Доказательство:}

Рассмотрим $\min k$, т.ч. среди первых $2k$ скобок поровну открывающих и закрывающих 

$\underset{1}{(}\underbrace{\ldots\ldots\ldots\ldots\ldots}_{\text{ПСП длины }2(k - 1)}\underset{2k}{)}\ \underbrace{\underset{2k + 1}{(} \ldots \underset{2n}{)}}_{\text{ПСП длины }2(n - k)}$

$C_{k - 1} \cdot C_{n - k},\ k = 1, 2, \ldots, n$

$C_n = \sum\limits_{k = 1}^n C_{k - 1}C_{n - k}$

\begin{nota}{}
    $C_0 = 1;\ C_n = C_0C_{n - 1} + C_1C_{n - 2} + \ldots + C_{n - 1}C_0\ n \geq 0$

    $C(t) = \sum\limits_{n = 0}^\infty C_nt^n;\ C_{n + 1}t^{n + 1} = (C_0C_n + C_1C_{n - 1} + \ldots + C_nC_0)t^{n + 1}$

    $\sum\limits_{n = 0}^\infty : C(t) - 1 = tC(t)^2 \Leftrightarrow t C(t)^2 - C(t) + 1 = 0$

    $C(t) = \frac{1 \pm \sqrt{1 - 4t}}{2t}$, т.к. $C(0) = 1$, то берем $C(t) = \frac{1 - \sqrt{1 - 4t}}{2t}$

    $\sqrt{1 - 4t} = (1 - 4t)^\frac{1}{2} = \sum\limits_{n = 0}^\infty (-1)^k {\frac{1}{2} \choose k}4^kt^k = 1 + \sum\limits_{k = 1}^\infty \frac{(\frac{1}{2})(- \frac{1}{2}) \ldots (\frac{3 - 2k}{2})}{k!}(-1)^k2^{2^k}t^k = \\
    = 1 - \sum\limits_{k = 1}^\infty \frac{1 \cdot 3 \cdot 5 \cdot \ldots \cdot (2k - 3)}{k!}\underbrace{2^k}_{2 \cdot \frac{2 \cdot \ldots \cdot (2k - 2)}{(k - 1)!}}t^k = 1 - \sum\limits_{k = 1}^\infty \frac{2}{k} \frac{(2k - 2)!}{(k - 1)!^2}t^k = 1 - \sum\limits_{k = 1}^\infty \frac{2}{k} {2k - 2 \choose k - 1}t^k$

    $C(t) = \sum\limits_{k = 1}^\infty \frac{1}{k} {2k - 2 \choose k - 1}t^{k - 1} = \sum\limits_{n = 0}^\infty \frac{1}{n + 1}{2n \choose n}t^n$

    Мы доказали теорему $C_n = \frac{1}{n + 1}{2n \choose n}$
\end{nota}

\begin{theo}{}
    $C_n = \frac{1}{n + 1}{2n \choose n}$
\end{theo}

\newpage

\section{\S 4. Разбиение чисел}

\begin{nota}{Вопрос}
    Сколько способов разбить $n$ в сумму $n = x_1 + \ldots + x_k;\ k \in \N$?

    \begin{enumerate}
        \item Упорядоченное разбиение (т.е. порядок важен)
        
        \textbf{Ответ:} ${n - 1 \choose k - 1}$

        \item Неупорядоченное разбиение
    \end{enumerate}
\end{nota}

\begin{defin}{}
    $p_k(n)$ -- количество представление $n = x_1 + \ldots + x_k$, где $x_1 \geq x_2 \geq \ldots \geq x_k \geq 1$

    $n - k = \underbrace{(x_1 - 1)}_{y_1} + \underbrace{(x_2 - 1)}_{y_2} + \ldots + \underbrace{(x_k - 1)}_{y_k} = y_1 + \ldots + y_k$, где $y_1 \geq \ldots \geq y_k \geq 0$

    Пусть $y_s > 0, y_{s + 1} = \ldots = y_n = 0$ -- таких способок $p_s(n - k)$

    $p_k(n) = p_k(n - k) + \underbrace{p_{k - 1}(n - k) + \ldots + p_1(n - k)}_{p_{k - 1}(n - 1)} = p_k(n - k) + p_{k - 1}(n - 1)$
\end{defin}

\begin{Example}{}
    $p_2(n) = p_2(n - 2) + p_1(n - 1) = p_2(n - 2) + 1$

    $p_2(n) = \begin{cases}
        \frac{n}{2} & n \divby 2 \\
        \frac{n - 1}{2} & n \not\divby 2
    \end{cases}$

    $p_3(n) = \begin{cases}
        \frac{n^2}{12} & n \equiv 0 \pmod 6 \\
        \frac{n^2 - 1}{12} & n \equiv \pm 1 \pmod 6 \\
        \frac{n^2 - 4}{12} & n \equiv \pm 2 \pmod 6 \\
        \frac{n^2 + 3}{12} & n \equiv \pm 3 \pmod 6
    \end{cases}$
\end{Example}

\begin{lem}{}
    $P_j(n) := 1^j + \ldots + n^j$ -- многочлен степени $j + 1$ от $n$ со старшим коэффициентом $\frac{1}{j + 1}$
\end{lem}

\textit{Доказательство:}

$(a + 1)^{j + 1} - a^{j + 1} = {j + 1 \choose 1} a^j + {j + 1 \choose 2} a^{j - 1} + \ldots + {j + 1 \choose j} a + 1$

Воспользуемся для $a = 0, 1, 2, \ldots, n$

$(n + 1)^{j + 1} = {j + 1 \choose 1}P_j(n) + {j + 1 \choose 2}P_{j - 1}(n) + \ldots + {j + 1 \choose j}P_1(n) + \underbrace{P_0(n)}_n$

$P_j(n) = \frac{1}{j + 2}((n + 1)^{j + 1} - {j + 1 \choose 2}P_{j - 1}(n) - \ldots - {j + 1 \choose j}P_1(n) - n)$

\begin{theo}{}
    $p_k(n)$ -- многочлен, коэффициенты которого зависят от $n \mod k!$, старший член $\frac{n^{k - 1}}{(k - 1)!k!}$
\end{theo}

\newpage

\textit{Доказательство:}

Индукция по $k$. База: $k = 1, 2$ -- смотреть выше 

Переход: $k - 1 \to k$

$p_k(n) - p_k(n - k) = p_{k - 1}(n - 1) = \frac{(n - 1)^{k - 2}}{(k - 2)!(k - 1)!} + \ldots = \frac{n^{k - 2}}{(k - 2)!(k - 1)!} + R_{k - 3}^{(0)}(n)$, где $R_{k - 3}^{(0)}(n)$ -- многочлен от $n$ степени $\leq k - 3$, коэффициенты которого зависят от $n \mod (k - 1)!$

Аналогично: $p_k(n - jk) - p_k(n - (j + 1)k) = \frac{n^{k - 2}}{(k - 2)!(k - 1)!} + R_{k - 3}^{(j)}(n)$

$p_k(n) - p_k(n - k!) = \frac{n^{k - 2}}{(k - 2)!} + S_{k - 3}(n)$

$n = k! m + r$

$p_k(n) = \frac{\overbrace{(k! m + r)^{k - 2}}^{k!^{k - 2}m^{k - 2} + \ldots}}{(k - 2)!} + \frac{(k! (m - 1) + r)^{k - 2}}{(k - 2)!} + \ldots + \frac{(k! + r)^{k - 2}}{(k - 2)!} + S_{k - 3}(k! m + r) + S_{k - 3}(k! (m - 1) + r) + \ldots + \\ + S_{k - 3}(k! + r) + p_k(r)$

$\frac{k!^{k - 2}}{(k - 2)!} \underbrace{(m^{k - 2} + (m - 1)^{k - 2} + \ldots + 1)}_{\frac{m^{k - 1}}{k - 1} + \ldots} + (\text{многочлен степени } \leq k - 2) = \frac{k!^{k - 2}}{(k - 2)!} \cdot \frac{m^{k - 1}}{k - 1} + \ldots = \\ = \frac{(mk! + r)^{k - 1}}{(k - 1)!k!} + \ldots$

\begin{Exercise}{Заметки на полях}
    $Q(n + 1) - Q(n)$ -- многочлен степени $k$ от $n \Rightarrow Q(n)$ -- многочлен степени $k + 1$ от $n$
\end{Exercise}

\begin{theo}{Следствие}
    $p_k(n) \sim \frac{n^{k - 1}}{(k - 1)!k!}$
\end{theo}

\begin{Remark}{Сравнение с упорядоченным разбиением}
    $\frac{{n - 1 \choose k - 1}}{p_k(n)} \sim \frac{\frac{(n - 1)(n - 2)\ldots(n - k + 1)}{(k - 1)!}}{\frac{n^{k - 1}}{(k - 1)!k!}} \sim \frac{\frac{n^{k - 1}}{(k - 1)!}}{\frac{n^{k - 1}}{(k - 1)!k!}} \xrightarrow[n \to \infty]{} k!$
\end{Remark}

\begin{theo}{}
    $p_k(n)$ -- количество способов разбить $n$ в сумму нескольких слагаемых, т.ч. $\max = k$
\end{theo}

\textit{Доказательство:}

Симметрия \href{https://ru.wikipedia.org/wiki/%D0%94%D0%B8%D0%B0%D0%B3%D1%80%D0%B0%D0%BC%D0%BC%D0%B0_%D0%AE%D0%BD%D0%B3%D0%B0}{диаграмм Юнга}

\begin{nota}{Задача о размене}
    $a_n$ -- количество способов разменять $n$ рублей при номиналах монет $1, 2, 5, 10$

    $\A(t) = \sum\limits_{n = 0}^\infty a_nt^n;\ n = a + 2b + 5c + 10d$

    $\A(t) = (1 + t + t^2 + \ldots)(1 + t^2 + t^4 + \ldots)(1 + t^5 + t^{10} + \ldots)(1 + t^{10} + t^{20} + \ldots) = \\
    = \frac{1}{1 - t} \cdot \frac{1}{1 - t^2} \cdot \frac{1}{1 - t^5} \cdot \frac{1}{1 - t^{10}} = \frac{P(t)}{(1 - t^{10})^4}$, где $P(t) = ( 1 + t + t^2 + \ldots + t^9)(1 + t^2 + t^4 + t^6 + t^8)(1 + t^5)$ 

    $\frac{1}{(1 - t^{10})^4} = \sum\limits_{n = 0}^\infty {n + 3 \choose 3}t^{10n}$
\end{nota}

\begin{defin}{Количество разбиений}
    $H \subset N$. $p(H; n)$ -- количество способок разбить $n$ в сумму слагаемых из $H$

    $p_k(n) = p(\{1, 2, \ldots, k\}; n) - p(\{1, 2, \ldots, k - 1\}; n)$
\end{defin}

\begin{defin}{}
    $p(n) = p(\N; n)$

    Производящая функция: $\P_H(t) = \sum\limits_{n = 0}^\infty p(H; n)t^n = \prod\limits_{k \in H} \frac{1}{1 - t^k}$
\end{defin}

\begin{theo}{Формула Харди-Раманджана}
    $p(n) \sim \frac{1}{4n\sqrt{3}} e^{\pi \sqrt{\frac{2}{3}}\sqrt{n}}$
\end{theo}

\begin{theo}{Теорема Эйлера}
    Число разбиений $n$ на нечетные слагаемые = количеству разбиеный $n$ на различные слагаемые
\end{theo}

\textit{Доказательство:}

$\prod\limits_{k = 1}^\infty (1 + t^k)$ -- производящая функция для числа разбиений на различные слагаемые 

$\prod\limits_{k = 1}^\infty (1 + t^k) = \prod\limits_{k = 1}^\infty \frac{1 - t^{2k}}{1 - t^k} = \frac{\prod\limits_{k = 1}^\infty \frac{1}{1 - t^k}}{\prod\limits_{k = 1}^\infty \frac{1}{1 - t^{2k}}} = \prod\limits_{k \not\divby\ 2} \frac{1}{1 - t^k}$

\begin{theo}{Пентагональная теорема}
    $\frac{1}{p(t)} = \prod\limits_{k = 1}^\infty (1 - t^k) = \sum\limits_{q \in \Z} (-1)^q t^{\frac{3q^2 - q}{2}}$
\end{theo}



\end{document}